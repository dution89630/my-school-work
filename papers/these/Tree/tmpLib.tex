\anote{Should include this?? What to include here??}

As mentioned in the introduction a partial solution to the difficulty of concurrent programming
is to provide concurrent libraries of popular abstractions such as data structures.
Often these are finely tuned high performance libraries
providing the programmer with useful operations.
The implementations of such libraries are usually very complex
and difficult to understand, but are provided with a simple to use
interface for anyone to use.
By doing this experts can create algorithms providing strong guarantees
and good performance that can then be widely reused.

In a perfect world transactional memory would not need such libraries,
as one could just directly place a sequential implementation of an abstraction
directly in transactions and it would perform as well and provide the same
guarantees as a library provided by an expert.
Unfortunately this is not completely true.
Due to the general nature
of the synchronization required by STM, a sequential data structure implementation directly
placed in transactions will never perform as well as a highly tuned concurrent
(lock based or non-blocking) implementation of a data structure done by an expert
given that the expert can consider the specific synchronization needed to implement
that data structure safely.

This gives us the opportunity to design libraries designed specifically for use inside
of transactions that take into account the synchronization considerations
of STM.
By doing this we can provide a programmer using transactions with abstractions that perform closer
to highly optimized non-transactional ones.

A further advantage of providing efficient libraries in STM is given by the reusable nature of code implemented using
transactions, a programmer can use the operations of the abstractions along with his own code
to create new efficient atomic operations.
For example a programmer might use a map abstraction that provides an $\lit{insert}$ and a $\lit{delete}$ operation
to create an atomic $\lit{move}$ operation by simply combing the $\lit{insert}$ and $\lit{delete}$ in a transaction.

By providing a programmer with a large selection of transaction based implementations of commonly used abstractions
his life becomes much easier.
Also if these libraries provide good performance then STM will be a very attractive alternative to highly tuned
non-transactional abstractions as STM has the advantage that the programmer can reuse the libraries in his transactions to create new
atomic operations specific to his needs.

This chapter will focus on designing efficient data structures in transactional memory proving the map abstraction.