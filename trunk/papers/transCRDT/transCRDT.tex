


% ============================================================================
\documentclass[11pt,letterpaper]{article}
%\usepackage[T1]{fontenc}
%\usepackage[latin1]{inputenc}
\usepackage{epic,eepic,amsmath,latexsym,amssymb,color,amsthm}
\usepackage{ifthen,graphics,epsfig,fullpage} 
\usepackage[english]{babel} 
\bibliographystyle{plain}
\usepackage{times}


% =========================================================================
\newcommand{\Xomit}[1]{}
\newcommand{\ignore}[1]{}
% =========================================================================


\begin{document}

%-----------------------for square--------------------------------------------
\newlength {\squarewidth}
\renewenvironment {square}
{
\setlength {\squarewidth} {\linewidth}
\addtolength {\squarewidth} {-12pt}
\renewcommand{\baselinestretch}{0.75} \footnotesize
\begin {center}
\begin {tabular} {|c|} \hline
\begin {minipage} {\squarewidth}
\medskip
}{
\end {minipage}
\\ \hline
\end{tabular}
\end{center}
}  
 
%--------------------------------------------------------------------
%--------------------------------------------------------------------
%-------- macros for algorithm ---------------------------------------
\newtheorem{definition}{Definition}
\newtheorem{theorem}{Theorem}
\newtheorem{lemma}{Lemma}
\newtheorem{corollary}{Corollary}
\newcommand{\toto}{xxx}
\newenvironment{proofT}{\noindent{\bf
Proof }} {\hspace*{\fill}$\Box_{Theorem~\ref{\toto}}$\par\vspace{3mm}}
\newenvironment{proofL}{\noindent{\bf
Proof }} {\hspace*{\fill}$\Box_{Lemma~\ref{\toto}}$\par\vspace{3mm}}
\newenvironment{proofC}{\noindent{\bf
Proof }} {\hspace*{\fill}$\Box_{Corollary~\ref{\toto}}$\par\vspace{3mm}}


\newcounter{linecounter}
\newcommand{\linenumbering}{\ifthenelse{\value{linecounter}<10}
{(0\arabic{linecounter})}{(\arabic{linecounter})}}
\renewcommand{\line}[1]{\refstepcounter{linecounter}\label{#1}\linenumbering}
\newcommand{\resetline}[1]{\setcounter{linecounter}{0}#1}
\renewcommand{\thelinecounter}{\ifnum \value{linecounter} > 
9\else 0\fi \arabic{linecounter}}

\newcommand{\tuple}[1]{\ensuremath{\left \langle #1 \right \rangle }}

%----------------------------------------------------------------------

Originally proposed as a hardware mechanism for easily developing non-blocking data structures,
Transactional Memory (TM) has since evloved to become a general pourpose method for easily writing
efficient concurrent code.
With TM a programmer is able to declare atomic transacions, more specifically, blocks
one or several reads and writes to shared memory bounded by calls to the
$\ms{begin\_transaction}$ and $\ms{end\_transaction}$ operations, with
these reads and writes appearing to have executed in exclusion with no interleaving
between other transactions, as if a single global lock had been used.

Unlike using a global lock, several transactions can be executed at the same time,
with the system ensuring that they appear to have executed in exclusion.
In order to still ensure this, the TM system will keep track of reads and writes
performed by transactions, detecting conflicts, and aborting and restarting transactions
as necessary.
Defining precisely how transactions execute in relation to eachother is a consistency criterion.
Probably the most commonly used criterion, linearizability, roughly says that each
operation appears to have occured at some instant in time between its invocation and completion.
Opacity, the most common criterion used for TM, ensures this for both aborted and committed transactions.

While for certain applications, certain STM systems exhibit performance that scales as the number
of processing core increases, many applications 

\begin{figure}[htb]
\centering{ \fbox{
\begin{minipage}[t]{150mm}
\footnotesize 
\renewcommand{\baselinestretch}{2.5} 
%\resetline
%\setcounter{linecounter}{200}
\begin{tabbing}
aaaaaaa\=aa\=aaaaa\=aa\=aa\=\kill %~\\


{\bf operation}  ${\sf non\_transactional\_read}(\mathit{addr})$ {\bf is}\\
\line{A01} \> $\mathit{tmp} \gets {\sf load}(\mathit{addr})$; \\ 
\line{A02} \> {\bf if} $( ~\mathit{tmp}$ is of type T $ \wedge \mathit{tmp.status} \neq$ COMMITTED ) \\
\line{A03}  \>\>  {\bf then if}  $(\mathit{tmp.time}  \leq \mathit{time}  \wedge  \mathit{tmp.status} = $ LIVE) \\
\line{A04} \>\>\>\> {\bf then} \=${\sf \mathit{C\&S}}$($tmp.status$, LIVE, ABORTED) {\bf end if}; \\
%\line{A05} \>\> {\bf end if} \\
\line{A06} \>\>\> {\bf if} ($tmp.status \neq $ COMMITTED)  \\
\line{A07} \>\>\>\> {\bf then} $\mathit{value} \gets \mathit{tmp.last}$ \\
\line{A08} \>\>\>\> {\bf else} $\mathit{value} \gets \mathit{tmp.value}$ \\
\line{A05} \>\>\> {\bf end if}; \\
\line{A09} \>\> {\bf else} $\mathit{value} \gets \mathit{tmp.value}$ \\
\line{A09A} \> {\bf end if}; \\
\line{A10} \> $\mathit{time} \gets {\sf max}(\mathit{time}, \mathit{tmp.time})$ \\
\line{A11} \> {\bf if} ($\mathit{time} = \infty$) {\bf then} $\mathit{time} = \mathit{GCV}$ {\bf end if}; \\
\line{A12} \> ${\sf return}$ ($\mathit{value}$) \\
{\bf end operation}. \\
\\

{\bf operation}  ${\sf contains}(\mathit{key})$ {\bf is}\\
\line{A01} \> {\bf if} $key \in \mathit{list}$ {\bf then} \\
\line{A01} \>\> ${\sf return}$ ($\sf{true}$) {\bf end if} \\
\line{A01} \> $\mathit{hash} \gets {\sf hash}(\mathit{key})$ \\


\line{A01} \> $\mathit{tmp} \gets {\sf load}(\mathit{addr})$; \\ 
\line{A02} \> {\bf if} $( ~\mathit{tmp}$ is of type T $ \wedge \mathit{tmp.status} \neq$ COMMITTED ) \\
\line{A03}  \>\>  {\bf then if}  $(\mathit{tmp.time}  \leq \mathit{time}  \wedge  \mathit{tmp.status} = $ LIVE) \\
\line{A04} \>\>\>\> {\bf then} \=${\sf \mathit{C\&S}}$($tmp.status$, LIVE, ABORTED) {\bf end if}; \\
%\line{A05} \>\> {\bf end if} \\
\line{A06} \>\>\> {\bf if} ($tmp.status \neq $ COMMITTED)  \\
\line{A07} \>\>\>\> {\bf then} $\mathit{value} \gets \mathit{tmp.last}$ \\
\line{A08} \>\>\>\> {\bf else} $\mathit{value} \gets \mathit{tmp.value}$ \\
\line{A05} \>\>\> {\bf end if}; \\
\line{A09} \>\> {\bf else} $\mathit{value} \gets \mathit{tmp.value}$ \\
\line{A09A} \> {\bf end if}; \\
\line{A10} \> $\mathit{time} \gets {\sf max}(\mathit{time}, \mathit{tmp.time})$ \\
\line{A11} \> {\bf if} ($\mathit{time} = \infty$) {\bf then} $\mathit{time} = \mathit{GCV}$ {\bf end if}; \\
\line{A12} \> ${\sf return}$ ($\mathit{value}$) \\
{\bf end operation}. \\
\\

\end{tabbing}
\normalsize
\end{minipage}
}
\caption{Non-transactional operations for reading and writing a variable.}
\label{fig:ntops}
}
\end{figure}



\end{document}