% 
% % ============================================================================
% % ============================================================================
% \documentclass%[11pt]
% {elsarticle}
% \usepackage{epic,eepic,amsmath,latexsym,%fullpage,
% amssymb,color}
% \usepackage{ifthen,graphics,epsfig}
% \usepackage[english]{babel}
% \usepackage{times}
% %\usepackage{ulem}
% 
% \journal{Theoretical Computer Science}
% 
% \bibliographystyle{plain}
% 
% \begin{document}
% %-----------------------for square--------------------------------------------
% \newlength {\squarewidth}
% \renewenvironment {square}
% {
% \setlength {\squarewidth} {\linewidth}
% \addtolength {\squarewidth} {-12pt}
% \renewcommand{\baselinestretch}{0.75} \footnotesize
% \begin {center}
% \begin {tabular} {|c|} \hline
% \begin {minipage} {\squarewidth}
% \medskip
% }{
% \end {minipage}
% \\ \hline
% \end{tabular}
% \end{center}
% }  
%  
% %--------------------------------------------------------------------
% %--------------------------------------------------------------------
% %-------- macros for algorithm ---------------------------------------
% \newtheorem{definition}{Definition}
% \newtheorem{theorem}{Theorem}
% \newtheorem{lemma}{Lemma}
% \newtheorem{corollary}{Corollary}
% \newcommand{\toto}{xxx}
% \newenvironment{proofT}{\noindent{\bf
% Proof }} {\hspace*{\fill}$\Box_{Theorem~\ref{\toto}}$\par\vspace{3mm}}
% \newenvironment{proofL}{\noindent{\bf
% Proof }} {\hspace*{\fill}$\Box_{Lemma~\ref{\toto}}$\par\vspace{3mm}}
% \newenvironment{proofC}{\noindent{\bf
% Proof }} {\hspace*{\fill}$\Box_{Corollary~\ref{\toto}}$\par\vspace{3mm}}
% 
% 
% \newcounter{linecounter}
% \newcommand{\linenumbering}{\ifthenelse{\value{linecounter}<10}
% {(0\arabic{linecounter})}{(\arabic{linecounter})}}
% \renewcommand{\line}[1]{\refstepcounter{linecounter}\label{#1}\linenumbering}
% \newcommand{\resetline}[1]{\setcounter{linecounter}{0}#1}
% \renewcommand{\thelinecounter}{\ifnum \value{linecounter} > 
% 9\else 0\fi \arabic{linecounter}}
% 
% 
% 
% % ----------------------for appendix --------------------------------------
% \newenvironment{lemma-repeat}[1]{\begin{trivlist}
% \item[\hspace{\labelsep}{\bf\noindent Lemma~\ref{#1} }]}%
% {\end{trivlist}}
% 
% \newenvironment{theorem-repeat}[1]{\begin{trivlist}
% \item[\hspace{\labelsep}{\bf\noindent Theorem~\ref{#1} }]}%
% {\end{trivlist}}
% 
% \newenvironment{corollary-repeat}[1]{\begin{trivlist}
% \item[\hspace{\labelsep}{\bf\noindent Corollary~\ref{#1} }]}%
% {\end{trivlist}}
% 
% %==================================================================
% 
% % =========================================================================
% \newcommand{\Xomit}[1]{}
% %==========================================================================
% 
% 
% 
% \begin{frontmatter}
% %==========================================================================
% 
% %\title{\bf A different  look at software transactional memory systems
% 
% \title{Towards a universal construction for %\\
%            transaction-based multiprocess programs\tnoteref{title}}
% 
% %\tnotetext[eu]{This 
% %research is part  of the  Marie Curie ITN project TRANSFORM
% %funded by the European Union  FP7 Program (grant 238639).}
% \tnotetext[title]{A preliminary version of this paper has appeared in the proceedings of the 13th 
% International Conference on Distributed Computing and Networking (ICDCN 2012) \cite{CIR11}.}
% 
% %==========================================================================
% 
% %
% \author[irisa]{Tyler Crain}
% \author[irisa]{Damien   Imbs}
% \author[irisa,iuf]{Michel  Raynal\corref{cor1}}
% %        
% \address[irisa]{IRISA, Campus de Beaulieu, 35042 Rennes Cedex, France}
% 
% \address[iuf]{Institut Universitaire de France\\
% ${\tt \{tyler.crain|damien.imbs|raynal\}@irisa.fr}$}
% 
% \cortext[cor1]{Corresponding author}
% 
% 
% %\date{}
% %\maketitle
% %==========================================================================
% 
%   
% \begin{abstract}
% The aim of a Software Transactional  Memory (STM) system  is to discharge
% the programmer from the  explicit management of synchronization issues. 
% The  programmer's job resides in the design of multiprocess programs
% in which processes  are made up of transactions,  each transaction being an
% atomic execution unit that accesses  concurrent objects. 
% The important point is that the  programmer has to focus her/his efforts 
% only on   the parts of code which  have to be atomic  execution units  without
% worrying  on the way the corresponding synchronization has  to be realized. 
% 
% 
% 
% Non-trivial STM systems allow transactions to execute concurrently and 
% rely on the notion  of commit/abort of  a transaction in order to 
% solve their conflicts  on the objects they access simultaneously. 
% In some  cases, the management  of aborted transactions is left to the 
% programmer.   In other  cases,   
% the underlying system  scheduler  is  appropriately  modified 
% or an underlying contention manager is used in order 
% for each transaction to be  (``practically always'' 
% or  with high probability) eventually committed.
% 
% 
% This paper presents a deterministic STM system in  which (1) every 
% invocation of a  transaction is  executed  exactly once  and (2) the notion
% of commit/abort of a transaction remains  unknown to the programmer. 
% This system, which imposes restriction neither on the design of 
% processes  nor or their concurrency pattern, 
% can be seen as a  step towards  the design of a deterministic
% universal construction to execute transaction-based multiprocess 
% programs on top of a multiprocessor.  Interestingly, the proposed  
% construction  is lock-free (in the sense that it uses no lock). 
% \end{abstract}
% 
% \begin{keyword} 
% Abort/commit, Asynchronous system, Atomic execution unit, 
% Compare\&Swap, Concurrency management,  Fetch\&Increment, Lock-freedom,  
% Shared memory system, STM system, Transaction, Universal construction.
% \end{keyword} 
% %==========================================================================
% 
% \end{frontmatter}
% 


\section{Introduction}

The previous chapter focused on a few contributions to an area of transactional
memory research that takes a fixed view of the semantics of a transaction for the programmer
and studies what can be done in an STM protocol without changing the semantics.
This type of STM research puts first the ease-of-use for the programmer using the STM protocol
before considering secondary intrestes (usually performance).
In most cases this ease-of-use is ensured by having the protocol satisfy opacity.
Opacity is used because generally it is considered
to ensure the ammount of safety a programmer would expect from an atomic block without having
to worry about inconsistencies of aborted transactions
(Note that in some cases virtual world consistency is used as it does not change
how the programmer views a transaction).
This chapter takes a slightly differnt appproach to STM research as it suggest that transactional
memory might be more usable to a programmer if it satisfied more then just opacity.

Opacity and other consistency criteron are generally considered to be safety properties.
Informally this is because they ensure a protocol that implements them will
act in a way that a user would expect and not produce any weird behavoir.
For example opacity prevents any transaction from executing on invalid states of memory, preventing
things such as divide by zero exceptions in correct code.
What such consistency criteron do not consider (among other things) is how often transactions commit.
For example a protocol could satisfy opacity by just aborting every transaction before it performs
any action, but of course this protocol would be useless.
In order to avoid this, certain STM protocols satisfy liveness (or progress) properties
These properties, not limited to transactional memory, ensure the operations of a process
will make some sort of progress sometimes depending on the ammount of contention in the system.
Let us now look at some of these properties.

\subsection{Progress properties}
This section will give an overview of the most common progress properties defined
for concurrent algorithms.
They are arranged into two categories, blocking and non-blocking.

\subsubsection{Blocking properties}
The most common way to write concurrent programs is by using locks,
generally lock based programs satisfy blocking progress properties.
A property is blocking when a thread's progress can be blocked because it is waiting
for another thread to perform some actaion.
For example thread $A$ might want to accquire lock $l$, but thread $B$ currently owns lock $l$
so then thread $A$ waits for thread $B$ to release lock $l$.
In this case thread $A$ is blocked by thread $B$.
The three most common blocking properties are (from weakest to
strongest) \emph{deadlock freedom}, \emph{livelock freedom},
and \emph{starvation freedom}.
They are described breifly in the following paragraphs.

\paragraph{Deadlock freedom}
Deadlock freedom is the weakest blocking property, it prevents the implementing protocol
form entering a state of deadlock.
Deadlock occurs when at least two threads are preventing eachother from progressing
due to eachother holding a lock (or resourse) that the other wants to accquire.
A simple example of deadlock would be the following:
thread $A$ owns lock $l1$ and wants to accquire lock $l2$, concurrently
thread $B$ owns lock $l2$ and wants to accquire lock $l1$.
In this case thread $A$ and $B$ will be stuck infinately, wating to accquire the lock that the other
already owns creating a state of deadlock.
It is generally considered that every correct concurrent protocol should at least satisfy deadlock
freedom.
A common way to avoid deadlock freedom is by ensuring threads accquire locks in a fixed global order.
\paragraph{Livelock freedom}
Stronger then deadlock freedom, livelock freedom prevents a state of livelock in which threads
can never progress due to their progess depending on a shared state that is created by another thread.
Consider the follwing simple example:
in order to progress a thread must own locks $l1$ and $l2$ concurrently.
Thread $A$ runs a protocol that first acquires $l1$ and then $l2$, while thread
$B$ runs a protocol that first acquires $l2$ then $l1$.
In order to avoid deadlock when a thread noticies that another thread owns a lock it wants, it
relases all the locks it owns and starts the locking protocol over.
Consider the events of thread $A$ and $B$ happen in the following order:
$A.acquire(l1) = success$, $B.acquire(l2) = success$, $A.is\_owned(l2) = true$, $B.is\_owned(l1) = true$,
$A.release(l1)$, $B.release(l2)$.



\anote{NEED TO DEFINE THIS HISTORY STRUCTURE}


If such an order of events is continually reapeted then livelock is observed.
It should be noted that by definition livelock freedom also ensure deadlock freedom.
\paragraph{Starvation freedom}
The strongest blocking property, starvation freedom ensure that no threads starve.
A thread is starved when it requires access to some shared resourses in a certain state to progress, but
it is never able to such gain access to them due to one or more concurrent ``greedy'' threads
that consistantly own the needed resources.
Starvation freedom prevents both deadlock and livelock from happening.


\subsubsection{Non-blocking properties}
There are also several non-blocking progress properties of concurrent programming.
Unlike the blocking properties these ensure that a thread's progress is never blocked
due to it waiting for another thread.
Inherantly this means that it is not possible protocols that use standard locks
to be non-blocking.
Instead of using locks, non-blocking protocols generally use atomic operations
such as test-and-set or compare-and-swap (in a non-blocking fashion) when synchronization between threads is necessary.
Generally programming using these operations in this way is considered to be much more difficut then
programming using locks.

\paragraph{The problem with blocking}
If non-blocking algorithms are more difficult to program then why not just use locks?
When concerning scalibility, non-blocking algorithms have two main advantages over their blocking counterparts.

The first most obvious advantage is by definition, in a non-blocking algorithm a thread will never be blocked
waiting for another thread.
Normally waiting might seem to be necessary part of synchronization, in our everyday when working together
with someone on a project we will wait for a teammate to finish their task before starting ours.
But then consider a massive project that involves hundreds of people across the world in multiple organizations,
one person on this project might have an approching deadline and he might not want to wait on someone across the
world that he has never met before in order to start his task.
Similar situations can exist in largely parallel computer systems, threads might be spread across multiple
processors or machines, some might be sleeping, some might execute slower then others, the data connection between
some processors might be slower then others.
In such cases the ammount of time a thread might have to wait could be unknown and harmful to scalibility.

The second advantage is when faults are considered.
If a thread is waiting for a lock that is owned by another thread that has crashed
then without fault detection and recovery this thread will be waiting forever.
Non-blocking algorithms on the other hand by definition do not have to worry about this.
Given that fault detection and recovery is a difficult problem especially in massively parrallel
systems this is an obvious advantage of non-blocking algorithms.

\paragraph{Non-blocking}
The first non-blocking property is \emph{obstruction-freedom}.
A protocol that is obstruction-free ensures that any thread will eventually make progress
as long as it is able to run by itself for long enough.
This property ensure that no thread is ever stuck wating for another thread, but only guarantees
progress in the absence of contention.

\paragraph{Lock-freedom}
For certain tasks progress might be required even in the face of contention, in such cases
non-blocking is not strong enough.
\emph{Lock-freedom} ensures that at any time there is at least one thread who will enventually make progress.
Even though some threads may starve, in a lock-free algorithm we at least know that the
system as a whole is making progress.

\paragraph{Wait-freedom}
An even stronger progress property \emph{wait-freedom} ensures that
all threads eventually make progress.
This is a nice property to ensure as each thread in the system
is only dependent on itself and not other threads for making progress.










\subsection{Universal Constructions for concurrent objects}
\label{sec:univ-const}


Given the increased progress guarenteed by wait-free protocols
they are desireable over lock based protocols.
Unfortunately wait-free protocols are
known to be extreemly difficult to write and understand.

Two years before the concept of transactional memory 
was introduced, the notion of a universal construction for concurrent objects (or 
concurrent data  structures) was introduced by Herlihy \cite{H91}.

Like transactional memory, a universal construction's main concern is
with making concurrent programming easier.
A universal construction takes any sequential implementation
of an object or data structure and makes its operations concurrent, wait-free,
and linearizable.
The concurrent objects suited to such   constructions are 
the objects that are  defined by a sequential specification
on total operations (i.e., operations that, when executed alone, 
always return a result).
For example data structures are the typical example
of the use of a universal construction.

%
%  MOVED IN THE DISCUSSION IN FIRST PART OF THE CONCLUSION    
%There is an important and fundamental difference
% between an operation on a concurrent 
%object (e.g., a shared queue) and a transaction. 
%Albeit the operations on a queue can have many different implementations, 
%their semantics is  defined once for all. 
%Differently, each transaction is a specific  atomic procedure whose 
%code can be  seen as being any dynamically  defined code.
%


\paragraph{A brief introduction to a universal construction protocol}
Upon first inspection it might appear to be a nearly impossible task to design
a construction that can automatically turn the operations of a sequential
object into a concurrent one with such a strong progress guarantee as wait-freedom.
Fortunately even though the fine details of the universal construction proposed
in \cite{H91} might be intracate, the key design concepts are quite clear.

The first concept has to deal with correctness.
How to ensure that the sequentail code is exectued safely when there
can be multiple threads concurrently performing operations on the object?
This is ensured simply by each thread operating on a local copy of the object.
Before a thread starts executing the original sequential code, it makes a copy
of the object in its local memory and the operation is performed on that object.
Once the sequential operation is complete it must be then made visible so other
threads can be aware of the modificaiton.
In order to acheive this there is a single global operation pointer.
An operation completes by performing a compare and swap on this pointer
changing it to point to a descriptor of its operation.
The value swapped out must be the same as it was when the operation started,
if not the operation discards its modifications and starts over with a
new up to date local copy.
Unfortunately this means that best case performance will be no better
then a single thread, but in certain cases this might be an acceptable
trade-off for wait-free progress.

The second key concept has to deal with liveness.
As described in the previous paragraph an operation completes by
modifying a global pointer with a compare-and-swap operation,
but this operation can fail due to a concurrent modifcation
to the global pointer by some other thread.
Now according to the progress guaratee of wait-freedom, every operation
by every thread must eventually complete successfully without blocking,
meaning the compare-and-swap must not fail infinately many times.
The key concept used to ensure this is helping.
Since a failed compare-and-swap can only be caused by a different thread
succeeding with its compare-and-swap, why not have this successful
thread help the thread that failed?
Simply put helping here means that several threads will all execute
the operation of a thread whos operation has failed in order to ensure
that that operation eventually succeeds.
When helping it is important to ensure
that each operation is not performed several times.

\paragraph{Alternative universal constructions}
Since the original, several  universal constructions have been proposed
(e.g., \cite{ADT95,AM99,FK09}) focusing on ensuring different properties
or increased efficiency.
Interstingly many of the key design concepts from the original universal construction
such as helping are also included in these designs.

One intresting example related to the work in this thesis is
a universal construction for wait-free {\it transaction friendly} 
concurrent objects presented in \cite{CER10}. The words 
``transaction friendly'' means here that a process that has invoked an
operation on an object can abort it during its execution. Hence, 
a `transaction friendly'' concurrent object is a kind of  
abortable object.  It is important to notice  that  this  abortion 
notion is different from  the notion of  transaction abortion. 
In the first case, the abort of an operation is a programming level notion
that  the construction has to implement. 
Differently, in the second case,  a transaction abort is  due 
the implementation itself. More precisely, 
transaction abortion is then a system level mechanism used to  prevent  
global inconsistency when  the system allows concurrent transactions 
to be  executed
optimistically (differently, albeit very inefficient,  using a single 
global lock for  all transactions would allow any  transaction to executed
without being aborted).

(A main issue solved in  \cite{CER10}
lies in ensuring that, without violating wait-freedom,  an  operation $op$ 
issued by a process $p_i$  is not committed in the back of $p_i$ by another  
process $p_j$  which is  helping it to execute that  operation.) 


\subsection{Transactional memory and universal constructions}
The previous sections fist discussed progress properties for
concurrent code in general followed by a discussion
on universal constructions which can be used to create
a concurrent wait-free construction of any sequential object.
The following sections will discuss how progress properties
can be considered in transactional memory as well as the relation
between univeral constructions and transactional memory.

\subsection{Progress properties and transactional memory}
NOTE: That most STM don't ensure progress just for speed's sake.

The previously mentioned blocking and non-blocking progress
properties were defined for concurrent code in general
and do not concern the specifics of transactional memory.
The key difference to consider between transactional memory traditional concurrent
code is that transactions can abort and restart.
Does an aborted transaction entitle progress?
If we are just considering that code being executed entitles
progress then yes, but when considering the ease-of-use of transactional
memory it is more intersting to consider that only committed transactions
create progress.
The question of what to do with aborted transactions is an important one,
the following section first looks at how the previously mentioned progress
properties can be applied to transactional memory
followed by a breif overview of how progress is approached in transactional
memory.

\paragraph{How blocking and non-blocking progress properties relate to transactional memory}
Aborted transactions are not mentioned specifally in any of the blocking
or non-blocking progress properties.
Without considering aborted transactions it might not be very interesting
to have a transactional memory protocol that satisfies one of them
as the protocol could still just abort every transaction, being completely
useless to a programmer using the protocol.

We can then simply extend these properties by adding additional requirements
for the committing of transactions.
For example we might want a lock-free transactional memory protocol
to ensure that at least one of the live transactions in the system will
eventually commit.
A more detailed analysis of non-blocking progress properties and
transactional memory has been performed in \cite{}.
In this work they define the non-blocking properties
\emph{solo-progress}
as a equivalent to the obstruction-freedom property
for transactional memory and
\emph{local-progress} as a equivalent
of wait-freedome for transactional memory.
Informally a protocal that satisfies solo-progress must ensure
that every process they executes for long enough
must make progress (where progress requires eventually committing
some live transaction) while a protocol satisfying
local-progress must ensure that
``every process that keeps
executing a transaction (say keeps retrying it in case it aborts)
eventually commits it.''
Additionaly in \cite{} they examine how these properties can be appiled
in faulty and faut-free systems with or without parisitic transactions
(a parisitic transaction is one which is continually executed, but never
tries to commit).
They show that local-progess is impossible in a faulty system
where each transaction is fixed to a certain process.
An extended discussion on the possible/impossible liveness properties 
of a STM system is presented by the same authors in \cite{GK10} where it is 
also described  a general  lock-free STM system. This  system is 
based on a mechanism similar to the Compare\&Swap used in this paper.

\subsection{Previous approaches}
In order to ensure levels of progress and cope with aborted transactions,
several solutions have been proposed witch each taking differnt approches
to progress.
A whole range of solutions has been proposed.
Some do not directly confront the problem of progress,
focusing mainly on the performance of the protcols, while
others  offer   ``best effort semantics'' 
(which  means that there is no provable  strong guarantee)
and others offer provable guarantess of progress.

\paragraph{Programmer's task}
The least complex solution simply leaves the  management of aborted transactions
to the application programmer (similarly to  exception handling
encountered  in  some systems).
In such systems the programmer has the choice to have the protocol execute 
a chosen set of code when a transaction
is aborted one or several times.
This can be a powerful option for an experience programmer who knows the details
of an transactional memory implemntation, but this thesis takes the view
that the promary goal of transactional memory is ease-of-use and having programmers
have to manage aborted transactions themselves goes against this goal.


\paragraph{Contention Management} The idea is here to keep
track of conflicts between transactions and have a separate entity, usually
called {\it contention manager}, decide what action to take (if any). 
Some  of these  actions include  aborting one  or both  of  the conflicting
transactions, stalling one of the transactions, or doing nothing. 
The idea  was  first (as far as we know) proposed  in  the dynamic 
STM system (called DSTM) \cite{HLMS03} and much research has been done on the
topic since then. 

In some cases the contention manager's goal is to improve performance
while others
ensure (best effort or provable) progress guarantees or a combination. 
An associated theory is described in \cite{GHP05}. 
Failure detector-based contention managers (and corresponding 
lower bounds) are described in \cite{GKK08}. 
A construction to execute  parallel programs made up of {\it atomic blocks} 
that have to be dispatched to queues accessed by threads (logical processors) 
is presented in \cite{WF10}. 

An overview  of different contention managers and their performance is
presented  in \cite{GHP05}. Interestingly the authors find that there is no 
"best" contention manager and that the performance depends on the application. 
The notion of a {\it greedy contention manager} they propose
ensures that every issued transaction  eventually commits. 
This  is done  by giving  each transaction  a time-stamp  when it  is first
issued and,  once the  time-stamp reaches a  certain age, the  system ensures
that no  other transaction can commit  that will cause  this transaction to
abort.  Similarly to  the ``Wait/Die''  or ``Wound/Wait''
strategies used to solve deadlocks in some database systems  \cite{RSL78}, 
preventing  transactions from committing is  achieved by either
aborting them or directing them to wait. 
By  doing  this it  is  obvious  that  processes  with  conflicting
transactions  make progress.
In these blocking solutions, a transaction's eventual commit depends
on both on the  process that issued this transaction as well as the process
that issued the transaction  with the oldest time-stamp.

Unfortunately even though such contention managers exist that ensure
all transactions commit, most
STM implementations do not use them in the interest of performance
and as a result provide less strong progress guarentees.
As a solution to avoid these performance problems while still
eventually providing strong progress,  some modern  STM's (e.g., for example  
TinySTM  \cite{FFR08}   or  SwissSTM   \cite{DGK09})  use   less  expensive
contention  management  until  a transaction  has been  aborted a  certain
number  of times  at  which point greedy contention management is used. 

\paragraph{Transactional Scheduling}
Another approach to dealing with aborts consists in designing  schedulers that decide
when and how transactions are executed in the system base on certain
properties.
One approach is to design schedules that perform 
particularly well in appropriate workloads, for example the case of
read-dominated workloads is deeply investigated in \cite{AM09}.

Another interesting approach is
called  {\it steal-on-abort} \cite{ALKJKW09}. 
Its  base principle is the  following one.  If  a transaction $T1$
is  aborted  due to  a  conflict with a transaction  $T2$,  $T1$ is
assigned to  the processor  that executed  $T2$ in order  to prevent  
a new conflict between   $T1$ and  $T2$. 
Interestingly in order to help a transaction commit,  
this scheduler allows a transaction to be executed and committed
by a processor different  the one it originated from. 
%The idea  to execute  a transaction on   a different processor  has already
%been proposed  in \cite{ALKJKW09} and  \cite{AM09} where this idea is used
%to   provide processes with  a   type  of
%contention management refereed to as {\it transactional scheduling}. 
Like  contention  managers, these  schedulers  can  provide
progress,  but  none of  them  ensure  the progress  of  a  process with  a
transaction that conflicts with some  other transaction which has reached a
point at which it  must not be aborted. This means that  the  progress of a
process still  depends on the progress  of another process. 


\paragraph{Irrevocable Transactions} 
The  aim  of  the  concept   of  {\it  irrevocable}  (or  {\it  inevitable})
transaction  is  to   provide  the   programmer  with  a  special 
transaction type (or tag) related to its liveness or progress. 
%
Ensuring  that  a  transaction  does  not abort  is  usually  required  for
transactions  
that perform some operations that cannot be rolled back or aborted such as I/O.
In order to solve this issue, certain STM systems provide irrevocable 
transactions which will never be aborted once they are typed  irrevocable.
This is done by preventing concurrent conflicting transactions from 
committing when an irrevocable transaction is being executed.

It is interesting to note that (a) an irrevocable transaction must be run 
exactly  once and (b) only one irrevocable transaction can be executed at 
a time in the system (unless the shared memory accesses of the transaction 
are known ahead of time).
This priority given to the running irrevocable transaction allows it
to guarantee to succed, but does so at the cost of preventing other
transactions from progressing until it finishes.
STM protocols supporting ireovokable transactions are proposed and discussed  in 
\cite{SSDMS08}  and   \cite{WSA08}.  Irrevocable  transactions   suited  to
deadline-aware scheduling are presented  in \cite{MMFLMR11}. 

% The STM system proposed in this paper does not support 
% irrevocable transactions.
% Since  an irrevocable   transaction cannot  be executed  more than  once,
% it is obvious that irrevocable  transactions cannot  benefit from  the  
% helping  mechanism  proposed in  the paper. 

%
%%suppresed to comply with remark of referee 3 
%
%Our STM construction  could nevertheless be extended to provide this 
%functionality  by having  a process  that  wants to  execute an  irrevocable
%transaction issue a Compare\&Swap on an irrevocability flag to the end  
%of the list in order to  prevent any other concurrent  transaction from
%committing until  it commits itself. Unfortunately,  this  would violate the
%liveness and progress properties previously guaranteed. 

% That is why in our
% model,  ``irrevocable transactions'' appear as non-transactional code. 



\paragraph{Robust STMs} 
Ensuring progress even when bad behavior (such as process crash) can occur 
has been investigated in several papers. As an example,  \cite{WRFF10} 
presents a robust STM system where a transaction that is not committed 
for a too long period eventually gets  priority using locks. It is assumed 
that  the system   provides a   crash detection  mechanism that  allows 
locks to be stolen once a  crash  is detected. This paper also presents a
technique to deal with  non-terminating transactions. 
%Global  progress is also ensured when a transaction 
%executes an infinite loop. This is done  by aborting a suspected looping 
%transaction and then  executing it in isolation.  If the transaction  
%completes while in isolation it is then  retried in the normal mode.  


\paragraph{Obstruction-Freedom, Lock-Freedom} 
There have been several proposals for non-blocking STM protocols,
some of them  are obstruction-free  (e.g., \cite{HLMS03,ST97}),
while others are lock-free  (in the sense there is no deadlock) \cite{GHS08}.

In an obstruction-free STM system a transaction that is executed alone 
must eventually commit. So consider some transaction that is always 
stalled (before its commit operation) and, while it is stalled, 
some conflicting transaction commits. It is easy to build an execution 
in which this stalled transaction never commits.

In a lock-free STM system, infinitely many transaction invocations must 
commit   in  an infinite  execution.  Again it  is  possible  to build   an
execution in which a transaction is always stalled  
(before its commit operation) and is aborted by a concurrent transaction  
(transactions cannot wait for this stalled transaction because they do not 
know if it is making progress).

Unfortunately, none of them  provides the property that every issued 
transaction is committed.
As described in the previous section,
in order for the described universal constructions ensure that each operation
is performed succesfully, threads must help other threads by executing
eachother's operations.
In previously proposed non-blocking STM, helping only occurs
with transactions that are in the process of committing.
With only this type of help,  some transaction 
can be aborted indefinitely without violating safety.
Consider for example a thread $T1$ transaction $t1$
and a seperate thread $T2$ that executes an infinate sequence of the
same transaction $t2$.
Now simply consider a history as follows that is repeated infiantely, $t2$ starts executing,
$t1$ starts executing, $t2$ commits, $t1$ noticies that it conflicts with
$t2$ so it must abort.
In such a history $t1$ will always abort before it reaches the commitial phase
so if blocking is not allowed, helping only in the commital phase will not
ensure the commitial of every transaction.


\subsection{Ensuring transaction completion}
As seen, there are many ways to cope with aborted transactions
and to ensure progress in transactional memory.
Unfortunately none of these solutions quite realize to goal of ease-of-use for the
programmer is still concerned with the idea of abort/commit.
In some cases the programmer has to deal directly with aborted transactions
in his code, in others a programmer can prioritize certain transactions
so they will not abort, others allow transactions to be blocked, while
other allow transactions to be aborted infinately.
Absent from these solutions is the case where all transactions are guaranteed to
commit where the progress of a transaction does not rely on the
other processes then the one that issued the transaction.
Such a solution would prioritize ease-of-use as such a protocol would hide
the concept of aborted transactions from the programmer.

More precisely we want a non-blocking STM protocol which ensures that every transaction
issued by a process is eventually committed wheres its progress only depends
on the issuing process.
Then, the job of a
programmer is to write her/his concurrent program in terms of cooperating sequential
processes, each process being made up of a sequence of transactions (plus possibly
some non-transactional code). At the programming level, any transaction invoked by
a process is executed exactly once (similarly to a procedure invocation in sequential
computing). Moreover, from a global point of view, any execution of the concurrent
program is linearizable \cite{HW90}, meaning that all the transactions appear as if they have
been executed one after the other in an order compatible with their real-time occurrence
order. Hence, from the programmer point of view, the progress condition associated
with an execution is a very classical one, namely, starvation-freedom.


The remainder of this chapter focuses on the design of a such protocol that.
(As a note it should be said that the presentation of previous approches
that deal with aborted transactions
is not entirely fair as they are primarily
efficiency-oriented while our construction is
more theory-oriented.)


\subsection{A short comparison with object-oriented universal construction}
\label{sec:comparison-Herlihy}

The first point to notice is that an STM that hides the concept of
commit/abort from the programmer has objectives similar to that of a
universal consturction described earier in this chapter.
A universal construction allows a programmer to turn a sequential
object into a concurrent one where each operation is linearizable
and completes successfully, while
the STM protocol we want here is one that allows a programmer to place transactions
in his code each of which are executed successfully and are lineariable.
Although similar, the key difference between these lies in the
difference between an operation and a transaction.

% This universal  construction (denoted H\_UC in the following) 
% addresses  the  implementation  of linearizable concurrent objects in 
% presence  of process crashes. 

%As indicated in Section~\ref{sec:related-work},
There is an important and fundamental difference between an operation on 
a concurrent object (e.g.,  a shared queue) and a transaction performed
by an STM protocol. 
\anote{Clean up this}
Albeit the operations on a queue can have many different implementations, 
their semantics is  defined once for all. 
Differently, each transaction is a specific  atomic procedure whose 
code can be  seen as being any dynamically  defined code.
What is significant here is that any transaction is  able to read and
write to  any location in shared  memory while an operation  in a universal
construction is fixed to a predefined set, which is often a single instance
of a  data structure.
Simply put, a programmer might want to use a universal construction when
he has an object with predefined self-contained operations that
he wants to use concurrently (such as a data structure) while transactions
might be more suitable when the programmer wants to perform general
atomic operations within his code.

To briefly examine how these differences can effect the implementation
of a protocol we will look at the universal construction proposed
by Herlihy in \cite{H91} (denoted H\_UC in the following).
Interestingly H\_UC could be used for STM programs simply  by
piecing together  all the shared objects objects into a single concurrent object 
$\mathit{TO}$,  and  considering   all  the   transactions   as  operations
on this   object 
$\mathit{TO}$.  This  brute force approach is not conceptually satisfying.  
When considering lock-based mechanisms,   it is like using a single lock 
on  the single  ``big'' object $\mathit{TO}$, instead of a lock per object. 
Moreover, as it  requires each operation on an object to make a copy of 
this object   (before  accessing it),  H\_UC would force each  operation to
copy the whole  shared memory, even it  it works on a very  small subset of
its  content.  This is not the case in the construction proposed
in this chapter, where
only  the specific locations accesed by a transaction needs to be copied. 
The space granularities required by  H\_UC (when applied to STM)  and 
the proposed construction do not belong to the same magnitude order.
Traditionally STM protocols commonly use a read and write set
with the pourpose of tracking the  locations the STM  has read so far as well as those
that will be modified upon commit, validating these sets
in order to ensure correctness.
In the case of the STM protocol presented in this chapter these read
and write sets have the additional benifit of saving us from making copies of
the entire shared memory.

Even given the differences,
the proposed STM protocol in this chapter borrows key ideas from previous
universal constructions such as helping and using a shared global pointer
that is modififed using a compare-and-swap in order to ensure
progress and correctness.

Another interesting feature of  the proposed construction
(which is specifically designed for STM programs)  
is the  systematic use of  speculative execution.
Even  if  efficiency is   not a  first class  requirement addressed  in our
work, the notion of a  speculative  execution  can  be  a  basis  for  future
work  on  universal construction  that will focus on efficiency. 
As discussed in section \ref{sec:univ-const}, a traditional universal construction can expect best
case performance to be equal to that of a sequental implementation.
While in the case of the STM's speculative execution
any number of transactions are able to execute concurrently (and successfully if no
conflict is  found) by performing  validations on their read  sets.
In some ways, the computation cost of performing validation can be seen as
a trade off in order to allow for higher concurrency.

\section{A universal construction for transaction based programs}

The following secations  present a  new STM construction  that,
in order to hide the notion of abort/commit from the programmer, ensures every  
transaction issued by a process is necessarily committed and  each process 
makes progress.
More specifcally it ensures linearizable X-ability.
X-ability, or exactly-once ability, was originally defined 
by Fr{\o}lund and Guerraoui in \cite{FG01}
as a correctness condition for replicated services such as
primary-backup.  In this model there are actions, such as
transactions, that cause some side effect.  For a service to satisfy
X-ability, every invoked action and its side effect must be observed as
if it had happened exactly once.  In order to ensure this, actions
might be executed multiple times by the underlying system.  Given this
requirement, X-ability concerns both correctness and liveness and can
complement concurrency correctness conditions.

To  our knowledge,  this  is the  first STM  system 
we  know of  to combine  these concepts in a realistic protocol.
Given the similarities between this STM based construction and universal
constructions as well as the differences between transactions and operations on objects
we define this protocol as a ``universal construction for transaction based programs''.


%=========================================================================
\section{Computation models}
\label{sec:models}

This section presents the programming model offered to the programmers
and  the underlying  multiprocessor model on top of which  the universal STM 
system is built. 

In order to build such a construction, the paper assumes an underlying 
multiprocessor where the processors  communicate through 
a shared memory that provides them with atomic read/write registers, 
compare\&swap registers and fetch\&increment registers. 

As we will see, the underlying multiprocessor system  consists of 
$m$  processors where each processor is in charge  of a  subset of  processes.
The  multiprocess program, defined  by the  programmer, is  made up  of $n$
processes  where each process is a separate thread of execution.
 We say that a processor {\it owns}
the corresponding processes in the sense that  it has the responsibility 
of their individual   progress.  Given that at  the implementation level
a  transaction may abort, the processor $P_x$ owning the corresponding 
process $p_i$ can  require the help of  the other  processors  
in order for the transaction to be eventually committed. 
The  implementation of this  helping mechanism  is at the core of the 
construction  (similarly to the helping mechanism used  to implement 
wait-free operations despite any number of process crashes \cite{H91}). 
As we  will  see, the  main technical difficulties  lie  in   ensuring that 
(1) the helping  mechanism  allows   a transaction to be committed  exactly
once and (2) each processor $P_x$ ensures the individual progress  of  each 
process $p_i$ that it owns. As we can see, from a global point of view,
the  $m$ processors have to    cooperate in order to ensure a correct
execution/simulation of the $n$  processes. 


%------------------------------------------------------------------
\subsection{The  user programming model}

The program written by the user is made up of $n$ sequential processes
denoted $p_1$, ..., $p_n$. Each process is a sequence of transactions 
in which two consecutive transactions can be separated by non-transactional 
code. Both transactions and  non-transactional code can access concurrent 
objects.

%\vspace{-0.15cm}
\paragraph{Transactions}
A transaction is the description of an atomic unit of computation 
(atomic procedure) that can access concurrent objects called $t$-objects. 
``Atomic'' means that (from the programmer's point of view) each 
invocation of a transaction appears as being executed 
instantaneously at a single point of  the time line (between its start event
and its end event) and no two transactions are executed at the same point 
of the time line.  It is  assumed that, when executed alone, 
any  transaction  invocation always terminates.  

%\vspace{-0.15cm}
\paragraph{Non-transactional code}
Non-transactional code is made up of statements for which the user does not
require  them to appear  as being  executed  as  a single  atomic computation
unit. This code usually contains input/output statements (if any). 
Non-transactional code can also access concurrent objects.  
These objects are called $nt$-objects.  

%\vspace{-0.15cm}
\paragraph{Concurrent objects}
Concurrent objects shared by processes (user level) are denoted with
small capital letters. It is assumed that  a concurrent object is either 
an  $nt$-object  or a  $t$-object (not both). Moreover, each 
concurrent object is assumed to be linearizable. 

The atomicity property associated with a transaction guarantees 
that all its accesses to $t$-objects appear as being executed atomically. 
As each concurrent object is linearizable (i.e., atomic), 
the atomicity power of a transaction is useless if the transaction only 
accesses  a single  $t$-object once. Hence encapsulating accesses to
concurrent objects in a single transaction is ``meaningful'' only 
if that transaction   accesses several objects or accesses the same object
several times (as in a Read/Modify/Write operation).  

As an example let us consider a concurrent queue (there are very 
efficient implementation of such an object, e.g., \cite{MS96}). 
If the queue is always  accessed independently of the other concurrent 
objects,  its accesses can be part of non-transactional code and 
this queue instance is  then an $nt$-object. 
Differently,  if the queue is used with other objects
(for example, when  moving an item from a queue to another queue) the 
corresponding accesses  have to  be encapsulated in  a transaction  and the
corresponding queue instances are then  $t$-objects. 

%\vspace{-0.15cm}
\paragraph{Semantics}
As already indicated the properties offered to the user are
(1)  linearizability  (safety) and 
(2) the fact that each transaction invocation entails  exactly one
execution of that transaction (liveness). 



%-------------------------------------------------------------------------
\subsection{The underlying system model}
The underlying system is made up of $m$ processors (simulators) 
denoted $P_1$, ..., $P_m$.  We assume $n \geq m$.
The processors communicate through  shared memory that consists  of 
single-writer/multi-reader (1WMR) atomic registers, 
compare\&swap registers and fetch\&increment registers. 

%\vspace{-0.1cm}
\paragraph{Notation} 
The  objects shared by the processors are denoted with capital italic letters.
The local  variables of a processor are denoted with small italic letters.

%\vspace{-0.1cm}
\paragraph{Compare\&swap register}
A compare\&swap register  $X$ is an atomic object that provides 
processors with  a single  operation
denoted  $X.{\sf Compare\&Swap}()$. This operation is a conditional 
write that returns a boolean value. Its behavior can be described by
 the following statement:
\\
{\bf operation}  $X.{\sf Compare\&Swap}(old,new)$:\\
\hspace*{1cm} $atomic\{$
{\bf if}  $X=old$ 
{\bf then} $X\leftarrow new;$  ${\sf return} (\mathit{true})$
{\bf else} ${\sf return} (\mathit{false})$ 
{\bf end if}. 
$\}$



\paragraph{Fetch\&increment register}
A fetch\&increment register  $X$ is an atomic object that provides 
processors with a  single  operation,
denoted  $X.{\sf Fetch\&Increment}()$, that  adds $1$ to $X$ and 
returns its new value. 


%=========================================================================
\section{A universal  construction for STM systems}
\label{sec:construction} 

This section describes the proposed universal construction. 
It first introduces  the control variables shared by the $m$ processors
and then describes the  construction. As already indicated, its 
design is based on simple principles: 
(1) each processor is assigned a subset of processes for which 
it is in charge  of  their individual progress; 
(2) when  a  processor does not succeed in executing and committing  
a transaction issued by a process it owns,  it requires help from the other
processors;  
(3) the state of the $t$-objects accessed by transactions is represented 
by a list that is shared by the processors (similarly to \cite{H91}). 



Without loss of generality, the proposed construction considers that 
the  concurrent objects  shared  by transactions ($t$-objects) are  atomic
read/write objects. Extending to more sophisticated linearizable 
concurrent objects is possible. We limit our presentation to atomic
read/write objects to keep it  simpler. 


In our universal construction, the STM controls entirely the transactions; this is different from 
what is usually assumed in STMs \cite{GK10}.



%=========================================================================
\subsection{Control variables shared  by the  processors}

This section  presents  the shared  variables used by  
the processors to execute the multiprocess program. 
Each processor also has local variables which will be described when presenting 
the  construction.   

\paragraph{Pointer notation} 
Some  variables  manipulated  by  processors are  pointers.  The  following
notation is associated with pointers. Let  $PT$   be a pointer variable.  
$\downarrow PT$ denotes the object pointed to by $PT$. let 
$OB$ be an object. $\uparrow OB$ denotes a  pointer to $OB$. 
Hence, $\uparrow (\downarrow PT) =PT$ and  $ \downarrow (\uparrow OB) = OB$.  





\paragraph{Process ownership}
Each processor $P_x$  is assigned a set of processes for which it has the
responsibility of ensuring individual progress. A process $p_i$ is 
assigned to a single processor.  We assume here a static  assignment. 
(It is possible to consider a dynamic process assignment. This would 
require an appropriate underlying scheduler. We do not consider  
such a possibility  here in order to keep the presentation simple.)

The process assignment is defined by   an array 
$\mathit{OWNED\_BY}[1..m]$  such that the entry $\mathit{OWNED\_BY}[x]$ 
contains the set of identities of the processes ``owned'' by  processor $P_x$. 
As we will  see below the owner $P_x$ of process $p_i$ can ask other 
processors to help it execute the last transaction issued by $p_i$. 

\paragraph{Representing the state of the $t$-objects}
As previously indicated, at the processor (simulation) level, the state of the
 $t$-objects of the program is represented  by a list  of descriptors such 
that each descriptor is associated with a transaction that has been committed.

$\mathit{FIRST}$ is a compare\&swap register containing  a pointer to the
first descriptor of the list. Initially $\mathit{FIRST}$ points to a  list 
containing a single descriptor associated with a fictitious transaction that
gives an  initial value to  each $t$-object. 
Let $\mathit{DESCR}$ be the descriptor of a (committed) transaction
$T$. It has  the following four fields.
\begin{itemize}
%\vspace{-0.2cm}
\item 
$\mathit{DESCR}.next$ and $\mathit{DESCR}.prev$ are pointers to the next 
and previous items of the list.
%
%\vspace{-0.2cm}
\item 
$\mathit{DESCR}.tid$ is the identity of $T$. It is a pair 
$\langle i,t\_sn \rangle $ where $i$ is the identity of the process 
that issued the transaction and  $t\_sn$ is its sequence number (among all
transactions issued by $p_i$). 
%
%\vspace{-0.2cm}
\item 
$\mathit{DESCR}.ws$ is a set of pairs $\langle${\sc x},$v \rangle$ 
stating that $T$  has written $v$ into the concurrent  object~{\sc x}. 
%
%\vspace{-0.2cm}
\item $\mathit{DESCR}.local\_state$ is the local state of the process $p_i$ 
just before the  execution of  the transaction or the non-transactional code that follows $T$ in the code of $p_i$.
\end{itemize}



\noindent
{\it Helping mechanism: the array $\mathit{LAST\_CMT}[1..m,1..n]$}~~
This array  is  such that $\mathit{LAST\_CMT}[x,i]$ contains the sequence   
number of process $p_i$'s  last committed transaction as known by processor
$P_x$.  
$\mathit{LAST\_CMT}[x,i]$ is  written only  by  $P_x$. Its initial  value is
$0$.    


\paragraph{Helping mechanism: logical time}
$\mathit{CLOCK}$ is an atomic fetch\&increment register initialized 
to $0$.  It is used by the  helping mechanism to associate a logical date
with a  transaction that has to be  helped. Dates define a total  order 
on these transactions. They are used to ensure 
that any helped  transaction is  eventually  committed. 



\noindent
{\it Helping mechanism: the array $\mathit{STATE}[1..n]$}~~
This array  is  such that $\mathit{STATE}[i]$ describes the current state 
of the execution (simulation)  of process $p_i$. It has four fields.  
\begin{itemize}
%\vspace{-0.2cm}
\item  
$\mathit{STATE}[i].tr\_sn$ is the sequence number of 
the next  transaction to be issued by $p_i$. 
%
%\vspace{-0.2cm}
\item  $\mathit{STATE}[i].local\_state$ contains the local state of $p_i$
immediately before the execution of its next transaction (whose sequence 
number is currently  kept  in $\mathit{STATE}[i].tr\_sn$). 
%
%\vspace{-0.2cm}
\item 
$\mathit{STATE}[i].help\_date$ is an integer (date) initialized to $+\infty$. 
The processor $P_x$ (owner of process $p_i$) sets 
$\mathit{STATE}[i].help\_date$ to  the next value  of $\mathit{CLOCK}$ when
it requires help from  the other processors  in order for the last transaction 
issued by $p_i$ to be eventually committed. 
%
%\vspace{-0.2cm}
\item  $\mathit{STATE}[i].last\_ptr$ contains a pointer to a
descriptor of the transaction list (its initial value is  $\mathit{FIRST}$). 
$\mathit{STATE}[i].last\_ptr=pt$  means that, if 
the transaction identified by  $\langle i,\mathit{STATE}[i].tr\_sn\rangle$ 
belongs  to  the  list  of  committed  transactions,  it appears in the transaction list after  the
transaction pointed to by $pt$.  
\end{itemize}

%--------------------------------------------------------------------------
\subsection{How  the $t$-objects and $nt$-objects are represented}
Let us remember that the $t$-objects  and  $nt$-objects are the objects 
accessed by  the processes of the application program.  
The $nt$-objects are directly implemented in the  memory shared by the 
processors and consequently their operations access directly that memory.  

Differently, the  values of the $t$-objects are kept in the $ws$ field 
of the  descriptors associated with committed transactions
(these descriptors define the list pointed to by $\mathit{FIRST}$). 
More precisely,  we have the following.
\begin{itemize}
%\vspace{-0.2cm}
\item
A  write of a value $v$ into a $t$-object {\sc X}  by a transaction 
appears as the  pair  $\langle${\sc X}$,v \rangle$ contained in the 
field $ws$ of the descriptor  that is added to the list when the
corresponding transaction is committed. 
%\vspace{-0.2cm}
\item
A read of a $t$-object {\sc X} by a transaction is implemented by 
scanning  downwards (from a fixed local pointer variable $current$ towards $\mathit{FIRST}$)
the  descriptor list until encountering 
the first pair $\langle${\sc X}$,v \rangle$, the value $v$ being then 
returned by the read operation. It is easy to see that the values read by a 
transaction are always mutually consistent (if the values $v$ and $v'$ are 
returned by the reads of {\sc X} and {\sc Y} issued by the same transaction, 
then the first  value read  was not overwritten when the second one was read). 
\end{itemize}


%--------------------------------------------------------------------------
\subsection{Behavior of a processor: initialization}
Initially a   processor $P_x$ executes the non-transactional  code (if any)
of each process $p_i$ it owns  
until $p_i$'s first transaction and then 
initializes accordingly the  atomic register
$\mathit{STATE}[i]$. 
Next  $P_x$ invokes  ${\sf select}(\mathit{OWNED\_BY}[x])$
that returns the identity of a process it owns, this value is then assigned
to $P_x$'s local  
variable  $my\_next\_proc$. $P_x$ also initializes local variables
whose role  will be explained later.
This is described in Figure~\ref{fig:initialization}. 


The function ${\sf select}(set)$ is {\it fair} in the following sense:
if it is invoked infinitely often with $i\in set$,  then $i$ is returned
infinitely often (this can be easily implemented).
Moreover, ${\sf select}(\emptyset)=\bot$.  



%==================================================================
\begin{figure}[h!]
\centering{ \fbox{
\begin{minipage}[t]{150mm}
\footnotesize 
\renewcommand{\baselinestretch}{2.5} 
\resetline
\begin{tabbing}
aaaaa\=aa\=aaaaa\=aa\=\kill %~\\


{\bf for each} $i \in \mathit{OWNED\_BY}[x]$ {\bf do}\\
%~~\%  --- Initialization  ---------------------------------------------------

\> execute $p_i$ until the beginning of  its first transaction; \\

\>   $\mathit{STATE}[i] \leftarrow 
       \langle 1, p_i\mbox{'s~current local~state},   
                            +\infty, \mathit{FIRST} \rangle$  \\

{\bf end for};\\

$my\_next\_proc \leftarrow  {\sf select}(\mathit{OWNED\_BY}[x])$;\\
$k1\_counter  \leftarrow 0$; 
$my\_last\_cmt$ is a pointer initialized to  $\mathit{FIRST}$. %\\

\end{tabbing}
\normalsize
\end{minipage}
}
\caption{Initialization  for processor $P_x$ ($1 \leq x \leq m$)}
\label{fig:initialization}
}
\end{figure}

%=================================================================

%----------------------------------------------------------------------
\subsection{Behavior of a processor: main body}
The behavior of a processor $P_x$ is described in 
Figure~\ref{fig:main-construction}. This  consists of a while loop
that   terminates   when all transactions  issued by the processes owned by
$P_x$ have  been successfully   executed. This  behavior can  be decomposed
into 4 parts. 



\paragraph{Select the next transaction to execute} 
(Lines \ref{BA01}-\ref{BA11}) 
Processor $P_x$  first reads (asynchronously) the current  progress of each
process  and  selects accordingly   a process (lines  \ref{BA01}-\ref{BA02}). 
The procedure ${\sf select\_next\_process}()$  (whose details will be explained
later) returns  the identity $i$ of the process for which $P_x$ has to 
execute the next transaction. This process  $p_i$ can be a process owned by
$P_x$ or a process whose owner $P_y$ requires the other processors to help it
execute its next transaction.


Next, $P_x$  initializes local  variables in order  to execute  $p_i$'s next
transaction in the appropriate correct context (lines \ref{BA03}-\ref{BA05}).
Before entering a speculative  execution of the transaction, 
$P_x$ first looks to see
if it has not yet been committed (lines  \ref{BA07}-\ref{BA11}). To that
end,  $P_x$  scans  the list   of  committed transactions.   Thanks to  the
pointer value kept in  $\mathit{STATE}[i].last\_ptr$, it is useless to
scan the list from the beginning:  
instead the scan may start from the transaction 
descriptor pointed to by $current = state[i].last\_ptr$. 
If $P_x$  discovers that the  transaction has been previously  committed it
sets the boolean $committed$ to $\mathit{true}$. 

It is  possible that, while the  transaction is not  committed, $P_x$ loops
forever in the list because the predicate $(\downarrow current).next=\bot$ 
is  never true. This  happens  when  new committed  transactions (different
from $P_x$'s transactions) are repeatedly and infinitely 
added to the list. The procedure  ${\sf prevent\_endless\_looping}()$
(line \ref{BA07}) is used to prevent such an infinite looping. 
Its details will be  explained  later. 



%==================================================================
\begin{figure}[h!]
\centering{ \fbox{
\begin{minipage}[t]{150mm}
\footnotesize 
\renewcommand{\baselinestretch}{2.5} 
\resetline
\begin{tabbing}
aaaaa\=aa\=aaaaa\=aa\=\kill %~\\



{\bf while} ($my\_next\_proc \neq \bot$)  {\bf do}\\


\> \%  --- Selection phase  ---------------------------------------------------------------------------------\\



\line{BA01} \> $state[1..n] \leftarrow 
           [\mathit{STATE}[1],\cdots,\mathit{STATE}[n]]$;\\

\line{BA02} \>  $i \leftarrow {\sf select\_next\_process}()$;\\


\line{BA03} \> $i\_local\_state \leftarrow state[i].local\_state;$ 
               $i\_tr\_sn \leftarrow state[i].tr\_sn;$\\


\line{BA04} \> $current  \leftarrow state[i].last\_ptr$;
               $committed \leftarrow \mathit{false}$;\\ 


\line{BA05} \>  $k2\_counter \leftarrow 0$; 
               $\mathit{after\_my\_last\_cmt} \leftarrow \mathit{false}$;\\


\line{BA06} \> 
      {\bf while} \= $\big(~((\downarrow current).next  \neq \bot)
                      ~\wedge~ (\neg committed)~\big)$   {\bf do}\\


\line{BA07} \> \>   ${\sf prevent\_endless\_looping}(i)$;\\ 


\line{BA08} \> \>  
    {\bf if} \= $((\downarrow current).tid  =\langle i,i\_tr\_sn \rangle )$\\


\line{BA09} \> \> \>  
    {\bf then}  $committed \leftarrow \mathit{true}$;
    $i\_local\_state \leftarrow (\downarrow current).local\_state$ \\


\line{BA08'} \> \>   {\bf end if}; \\


\line{BA10} \> \> $current \leftarrow (\downarrow current).next$\\

\line{BA11} \>     {\bf end while}; \\


\line{BA12} \> {\bf if} \= $(\neg committed)$  {\bf then}\\

\>\> \% --- Simulation phase  ----------------------------------------------------------------------------\\


\line{BA13} \>  \>
   execute the $i\_tr\_sn$-th transaction of $p_i$:  the value of
   {\sc x}.${\sf read}()$  is obtained \\
                
\line{BA14} \>  \>
       by scanning downwards the transaction list 
       (starting from   $current$);  \\ %%%$state\_ptr$);  \\

\line{BA15} \> \> $p_i$'s local variables are read from (written into) 
             $P_x$'s local memory (namely,   $i\_local\_state$);\\  


\line{BA16}\>\> The set of shared objects read by the current 
              transaction  are saved in  the set $\mathit{lrs}$;\\ 


\line{BA17}\>\>  The  pairs $\langle${\sc y},$v\rangle$ 
               such that the transaction issued  {\sc y}.${\sf write}(v)$
               are saved in the set $ws$; \\



\>\> \% --- Try to  commit phase  ------------------------------------------------------------------------ \\


\line{BA18} \>\> $overwritten \leftarrow \mathit{false}$; \\ 


\line{BA19} \> \>
      {\bf while} \= $\big(~(\downarrow current).next \neq \bot)
      ~\wedge~ (\neg committed)~\big)$   %~\wedge~ (\neg overwritten)~\big)$  
        {\bf do}\\


\line{BA20} \> \>\>   ${\sf prevent\_endless\_looping}(i)$;\\ 
 
\line{BA21} \> \>\>  $current \leftarrow (\downarrow current).next$;\\

\line{BA22} \> \>\> same as lines \ref{BA08} and  \ref{BA09};\\


\line{BA23} \> \> \> {\bf if} 
    $(\exists$ {\sc x} $\in \mathit{lrs}:
               ~ \langle${\sc x}$,-\rangle \in (\downarrow current).ws)$
    {\bf then}  $overwritten  \leftarrow \mathit{true}$  {\bf end if} \\


\line{BA24} \>\>  {\bf end while};\\


\line{BA25} \>\>  {\bf if} \= $(\neg committed \wedge
                    ( \neg overwritten \vee ws=\emptyset))$\\


\line{BA26} \>  \>\> {\bf then} \=
   allocate a new transaction descriptor $\mathit{DESCR}$;\\


\line{BA27} \> \> \>\> $\mathit{DESCR} \leftarrow   
\langle  \bot, current,$ $\langle i, i\_tr\_sn \rangle,$ 
                       $ws,i\_local\_state \rangle$;\\
 

\line{BA28} \> \> \>\> $committed \leftarrow 
 {\sf Compare\&Swap}((\downarrow current).next, 
                     \bot,\uparrow \mathit{DESCR})$;\\


\line{BA29} \> \> \>\> {\bf if} ($\neg committed$) 
   {\bf then} disallocate  $\mathit{DESCR}$ {\bf end if}\\



\line{BA30} \> \> {\bf end if}\\

\line{BA31} \> {\bf end if};\\

\line{BA32} \> {\bf if} $(committed)$  {\bf then}
            $\mathit{LAST\_CMT}[x,i] \leftarrow i\_tr\_sn$  {\bf end if};\\

\>\> \% --- End of transaction --------------------------------------------------------------------------- \\


\line{BA33} \> {\bf if} \= $(i\in \mathit{OWNED\_BY}[x])$  {\bf then} \\


\line{BA34} \> \>  {\bf if} \= $(\neg committed)$ \\

\line{BA35} \> \> \> 
     {\bf then} \= {\bf if} \= $(state[i].help\_date =+\infty)$ {\bf then} \\

\line{BA36}  \> \> \> \> \> 
    $helpdate \leftarrow   {\sf Fetch\&Incr}(\mathit{CLOCK})$;\\ 

\line{BA37}  \> \> \> \> \> 
$\mathit{STATE}[i]\leftarrow 
    \langle   state[i].tr\_sn, state[i].local\_state, helpdate,
                         state[i].last\_ptr \rangle$\\

\line{BA38} \> \> \> \> {\bf end if} \\

\line{BA39} \> \> \>  
{\bf else} \>  execute non-transactional code of $p_i$ (if any) 
   in the local context  $i\_local\_state$;\\

\line{BA40} \> \> \> \>  
{\bf if} \= (end of $p_i$'s code)     \\


\line{BA41} \> \> \>\> \>{\bf then} \= 
  $\mathit{OWNED\_BY}[x]\leftarrow    
           \mathit{OWNED\_BY}[x] \setminus \{i\}$\\


\line{BA42} \>\>\>\>\> {\bf else} \> 

   $\mathit{STATE}[i] \leftarrow 
      \langle i\_tr\_sn +1, i\_local\_state, +\infty, current \rangle$\\

\line{BA43} \>\>\>\>  {\bf end if};\\ 

\line{BA44} \>\>\>\> 
 $my\_last\_cmt \leftarrow   ~\uparrow \mathit{DESCR}$;

 $my\_next\_proc \leftarrow  {\sf select}(\mathit{OWNED\_BY}[x])$\\


\line{BA45} \> \>  {\bf end  if}\\

\line{BA46} \>   {\bf end if}\\ 


{\bf end while}.  %\\


\end{tabbing}
\normalsize
\end{minipage}
}
\caption{Algorithm for processor $P_x$ ($1 \leq x \leq m$)}
\label{fig:main-construction}
}
\end{figure}

%=================================================================


\paragraph{Speculative execution of the selected  transaction} 
(Lines \ref{BA12}-\ref{BA17}) 
The identity of the transaction selected by $P_x$ is 
$\langle i,  i\_tr\_sn \rangle$. 
If,  from $P_x$'s point of view, this transaction  is not committed, $P_x$ 
simulates locally its execution (lines \ref{BA13}-\ref{BA17}).
The set of concurrent $t$-objects read by $p_i$ is  saved in $P_x$'s local
set $\mathit{lrs}$, and the  pairs $\langle${\sc y},$v\rangle$ 
such that the transaction issued {\sc y}.${\sf write}(v)$ are saved in 
the local set $ws$. This is a transaction's speculative execution by $P_x$. 

\paragraph{Try to commit the transaction}
 (Lines \ref{BA18}-\ref{BA32}) Once $P_x$ has performed a speculative
 execution of $p_i$'s  last transaction, it tries to commit  it by adding it
to the descriptor list, but only if certain conditions are satisfied.
To that end, $P_x$  enters a loop (lines \ref{BA19}-\ref{BA24}). 
There are two reasons for not trying to commit the transaction.
%
\begin{itemize}
%\vspace{-0.2cm}
\item
The first is when  the transaction has already  been committed. 
If this is the case, the transaction  appears in the list of committed 
transactions (scanned by the pointer $current$, lines \ref{BA21}-\ref{BA22}).
%
%\vspace{-0.2cm}
\item
The second is when the transaction is an update transaction and it
has read a $t$-object that  has then been overwritten (by a committed 
transaction). This is  captured by the predicate at line \ref{BA23}.
\end{itemize}

Then, if (a) the  transaction has not yet been  committed 
(as far as $P_x$ knows) and (b1) no $t$-object read has been overwritten  or 
(b2) the transaction is read-only, then $P_x$ tries to
commit its speculative execution of this transaction (line \ref{BA25}).
To do this it first creates a new descriptor $\mathit{DESCR}$,
updates its fields  with the data obtained from its speculative execution
 (line \ref{BA27}) and then tries to add it to the list. 
%
To perform  the commit, $P_x$ issues ${\sf Compare\&Swap}
((\downarrow current).next,\bot,\uparrow \mathit{DESCR})$.
It is easy to see that this invocation succeeds if and only if $current$ 
points to the last  descriptor of the list of committed transactions 
(line \ref{BA28}).

Finally, if the transaction has been committed
$P_x$ updates  $\mathit{LAST\_CMT}[x,i]$  (line \ref{BA32}).


\paragraph{Use the ownership notion to ensure the progress of each process} 
(Lines \ref{BA33}-\ref{BA46}) 
The last part of the description of $P_x$'s behavior  concerns the 
case where  $P_x$ is the owner of the process $p_i$ that issued the 
current transaction selected by $P_x$ (determined at line \ref{BA03}). 
This means that $P_x$ is responsible for guaranteeing the individual 
progress of $p_i$.  There are two cases. 
%
\begin{itemize}
%\vspace{-0.2cm}
\item
If $committed$ is  equal to $\mathit{false}$, $P_x$ requires help 
from the other processors in order for $p_i$'s transaction to be eventually 
committed. To that end, it assigns (if not yet done) the next date value 
to $\mathit{STATE}[i].help\_date$  (lines \ref{BA35}-\ref{BA38}). 
Then, $P_x$ proceeds to  the next loop iteration.
(Let us observe that, in that case, $my\_next\_proc$ is not  modified.)
%
%\vspace{-0.2cm}
\item
Given that $P_x$ is responsible for  $p_i$'s progress, 
if $committed$ is  equal to $\mathit{true}$ then $P_x$ executes the  
non-transactional code
(if any) that appears after the transaction (line \ref{BA39}).
Next, if  $p_i$ has terminated (finished its execution), $i$ is 
suppressed from $\mathit{OWNED\_BY}[x]$
(line \ref{BA41}). Otherwise, $P_x$ updates $\mathit{STATE}[i]$ 
in order for it 
to contain the information required to execute the next transaction of $p_i$ 
(line \ref{BA42}). 
%
Finally, before re-entering the main loop, $P_x$ 
updates the pointer  $my\_last\_cmt$ (see below) and 
$my\_next\_proc$ in order to ensure the progress of the next 
process it owns (line \ref{BA44}). 
\end{itemize}

%----------------------------------------------------------------------
\subsection{Behavior of a processor:  starvation prevention}
\label{sec:starvation-prevention}

Any transaction issued by a process has to be eventually executed 
by  a processor and  committed. 
To that end, the helping mechanism introduced previously has 
 to be enriched so that  no processor either (a) permanently helps  only
 processes owned by other processors or (b) loops forever in an internal 
while loop (lines \ref{BA06}-\ref{BA11} or \ref{BA19}-\ref{BA24}).
The first issue is solved by  procedure ${\sf select\_next\_process}()$
while the second issue is solved by the procedure 
${\sf prevent\_endless\_looping}()$.  

Each of these  procedures  uses an  integer value (resp., $K1$ and $K2$) 
as  a threshold  on the length of execution periods. These periods 
are measured with counters (resp., $k1\_counter$ and $k2\_counter$). 
When one of these periods attains its threshold, the corresponding 
processor requires help for its pending  transaction. 
The values $K1$ and $K2$ can be arbitrary. 


\paragraph{The procedure  ${\sf select\_next\_process}()$}
This operation is described in Figure~\ref{fig:select-next-proc}.
It is invoked at line \ref{BA02} of the main loop and 
returns a process identity.  Its aim is to allow the invoking processor 
$P_x$  to  eventually make  progress for each of the processes it owns. 

The problem that can  occur is that a processor  $P_x$ 
can permanently help other processors execute and commit
transactions of the processes they own, while none of the processes 
owned by $P_x$ is  making progress.  To prevent this bad scenario  
from occurring, a  processor $P_x$ that does not  succeed in having  
its current transaction executed and committed for  a ``too long''
period,  requires help from the other processors. 






%==========================================================================
\begin{figure}[htb]
\centering{ \fbox{
\begin{minipage}[t]{150mm}
\footnotesize 
\renewcommand{\baselinestretch}{2.5} 
%\resetline
\setcounter{linecounter}{100}
\begin{tabbing}
aaaaaaa\=aaaaaaa\=aaaaa\=aa\=\kill %~\\

{\bf procedure} 
${\sf select\_next\_process}()$ ${\sf returns}$ (process id) =\\

\line{CA01} \> 
{\bf let} $set~=~ \{~ i~|~   
      (state[i].help\_date \neq +\infty) ~\wedge$\\

\> ~~~~~~~~~~~~~~~~~~~~~
       $~\big(~(\forall y:~ \mathit{LAST\_CMT}[y,i] < state[i].tr\_sn)~
         \vee~ (i\in \mathit{OWNED\_BY}[x])~\big)\}$; \\ 




\line{CA02} \> {\bf if} \= $(set = \emptyset)$  \\

\line{CA03} \> \> {\bf then} \= 
  $i  \leftarrow my\_next\_proc$;  $k1\_counter \leftarrow 0$\\


\line{CA04} \> \> {\bf else} \>
 $i \leftarrow \min(set)$  computed with respect to transaction help dates;\\


\line{CA05} \>\>\>  {\bf if} \=  $(i \in \mathit{OWNED\_BY}[x])$\\


\line{CA06} \>\>\>\> {\bf then} \=  $k1\_counter \leftarrow 0$\\


\line{CA07} \>\>\>\> {\bf else} \> $k1\_counter \leftarrow k1\_counter+1$;\\

 
\line{CA08} \>\>\>\>\>  {\bf if} \= ($k1\_counter \geq  K1$) \\


\line{CA09} \>\>\>\>\>\> {\bf then} \= 

                         {\bf let} $j~=~my\_next\_proc$;\\ 


\line{CA10} \>\>\>\>\>\>\> {\bf if} \= 
       $(state[j].help\_date = + \infty)$ \\


\line{CA11} \>\>\>\>\>\> \> \> {\bf then} \=
        $helpdate \leftarrow {\sf Fetch\&Incr}(\mathit{CLOCK})$; \\

\line{CA12} \>\>\>\>\>\> \>\> \>

$\mathit{STATE}[j]\leftarrow$\\

\line{CA12'} \>\>\>\>\>\> \>\> \> ~~~~
    $\langle state[j].tr\_sn, state[j].local\_state, helpdate, 
                                          state[j].last\_ptr \rangle$ \\

\line{CA13} \>\>\>\>\>\> \>{\bf end if}; \\

\line{CA14} \>\>\>\>\>\> $k1\_counter \leftarrow 0$\\


\line{CA15} \>\>\>\>\>  {\bf end if}\\

\line{CA16} \>\>\> {\bf end if}\\

\line{CA17}  \>   {\bf end if};\\ 

\line{CA18} \>   ${\sf return}(i)$.  %\\ 


\end{tabbing}
\normalsize
\end{minipage}
}
\caption{The procedure ${\sf select\_next\_process}()$}
\label{fig:select-next-proc}
}
\end{figure}
%========================================================================

This is realized as follows. $P_x$ first computes the set $set$ of 
processes $p_i$ for which help has been required (those are the processes
 whose help date is $\neq +\infty$) and, (as witnessed by the array 
$\mathit{LAST\_CMT}$) either no processor has yet publicized the fact that
their last transactions have been committed  or $p_i$ is owned by $P_x$
(line  \ref{CA01}). 
If $set$ is empty (no help is required),  ${\sf select\_next\_process}()$
returns the identity of the next process owned by $P_x$ (line \ref{CA03}). 
If $set\neq \emptyset$, there are processes to help and $P_x$ selects the
identity $i$ of the process with the oldest help date  (line \ref{CA04}).  
But before returning the identity $i$ (line \ref{CA18}), $P_x$ 
checks if it has been  waiting for a too long period before having its next 
transaction executed.  There are then two cases. 
%
\begin{itemize}
%\vspace{-0.2cm}
\item 
If $i \in \mathit{OWNED\_BY}[x]$,  $P_x$ has already required 
help for the process $p_i$ for which it strives to make progress. 
It then resets the counter $k1\_counter$ to $0$
and returns the identity $i$  (line \ref{CA06}). 
%
%\vspace{-0.2cm}
\item 
If $i \notin \mathit{OWNED\_BY}[x]$, $P_x$ first increases  $k1\_counter$ 
(line \ref{CA07}) and  checks if it  attains its threshold $K1$. If this is 
the case, the logical period of time is too long  (line \ref{CA09}) and  
consequently (if not yet done) $P_x$ requires help for the last 
transaction of the process $p_j$ (such that $my\_next\_proc=j$).
As we have seen, ''require help'' is done by 
assigning the next clock value to $\mathit{STATE}[j].help\_date$  
(lines \ref{CA09}-\ref{CA13}). In that case,  $P_x$  also  resets  
$k1\_counter$ to $0$  (line \ref{CA14}). 
\end{itemize}


% ADDEd for referee 2:
Let us remark that the  procedure  ${\sf  select\_next\_process}()$ 
implements a kind of aging mechanism, which is  similar the one used by 
some   schedulers  to prevent process  starvation. 



%==================================================================
\begin{figure}[htb]
\centering{ \fbox{
\begin{minipage}[t]{150mm}
\footnotesize 
\renewcommand{\baselinestretch}{2.5} 
%\resetline
\setcounter{linecounter}{200}
\begin{tabbing}
aaaaaaa\=aa\=aaaaa\=aa\=\kill %~\\

{\bf procedure}  ${\sf prevent\_endless\_looping}(i)$;\\ 




\line{DA01} \> {\bf if} \= $(i\in \mathit{OWNED\_BY}[x])$ {\bf then}\\

\line{DA02} \> \> {\bf if}  
($\mathit{current}$ has bypassed $my\_last\_cmt$) 
     {\bf then} $ k2\_counter \leftarrow k2\_counter + 1$ {\bf end if};  \\

\line{DA03} \>\> {\bf if} \= 
     \big($( k2\_counter > K2)~\wedge~ (state[i].help\_date=+\infty)$\big) \\

\line{DA04} \>\>\>  {\bf then} \=
 $helpdate \leftarrow   {\sf Fetch\&Incr}(\mathit{CLOCK})$;\\

\line{DA05} \>\>\>\>
$\mathit{STATE}[i]\leftarrow 
 \langle   state[i].tr\_sn, state[i].local\_state, helpdate, 
                                          state[i].last\_ptr \rangle$\\

\line{DA06} \>\> {\bf end if} \\


\line{DA07} \> {\bf end if}.  %\\

\end{tabbing}
\normalsize
\end{minipage}
}
\caption{Procedure  ${\sf prevent\_endless\_looping}()$}
\label{fig-prevent-looping}
}
\end{figure}
%=================================================================

\paragraph{The procedure ${\sf prevent\_endless\_looping}()$}
As indicated, the aim of this procedure, described in 
Figure~\ref{fig-prevent-looping}, is to prevent a processor $P_x$ from 
endless looping in an internal  while loop (lines \ref{BA05}-\ref{BA09} 
or \ref{BA17}-\ref{BA21}).
%

The time period considered starts at the last committed transaction 
issued  by a  process owned  by $P_x$. It  is  measured by  the  number of
transactions committed since then.  The beginning of this time period is
determined by  $P_x$'s local pointer $my\_last\_cmt$  
(which   is initialized to $\mathit{FIRST}$
and updated at  line~\ref{BA44} of the main loop after the last transaction
of a process  owned by $P_x$  has been committed.) 


The relevant time period is measured by processor $P_x$ with  its local 
variable  $k2\_counter$.  If the process $p_i$ currently selected by 
${\sf select\_next\_process}()$ is owned by $P_x$ (line \ref{DA01}),
then $P_x$ will require help for $p_i$ once this period attains $K2$
(lines \ref{DA03}-\ref{DA06}). In that way, the transaction 
issued by that process will be executed and committed by other processors
and (if not yet done) this will  allow $P_x$ to exit the while loop because
its  local  boolean   variable  $committed$  will then  become  true  (line
\ref{BA09} of  the main loop).  


%========================================================================
\section{Proof of the STM  construction}
\label{sec:proof}

Let $\mathit{PROG}$ be a transaction-based $n$-process concurrent program. 
The proof of the universal construction consists in showing that a  
simulation of $\mathit{PROG}$ by $m$ processors that execute the algorithms 
described in Figures \ref{fig:initialization}-\ref{fig-prevent-looping}
generates an execution of  $\mathit{PROG}$. 



\begin{lemma}
\label{lemma:help-commit-1}

Let $T$ be the  transaction invocation  with  the smallest help date
(among all the transaction invocations not yet committed
for which help has been required). Let $p_i$ be the process that issued $T$
and $P_y$ a processor.
If $T$ is never committed, there is a time after which $P_y$ issues an 
infinite number invocations of ${\sf select\_next\_process}()$ 
and they all return $i$. 
\end{lemma}

\begin{proofL}
Let us assume by contradiction that there is a time after which
either $P_y$ is blocked within an internal while loop 
(Figure~\ref{fig:main-construction})  or its
invocations of ${\sf select\_next\_process}()$   never  return $i$. 
It follows from line \ref{CA04} of ${\sf  select\_next\_process}()$  that
the process identity  of the transaction from $set$  with the smallest help
date is returned. 
This means  that for $i$  to never be  returned, there must always  be some 
transaction(s) in $set$ with a smaller help date than $T$. 
By  definition we know  that $T$  is the  uncommitted transaction  with the
smallest help date, so any transaction(s) in $set$ with a smaller help date
must be  already committed. 
Let us call  this subset of committed  transactions  $T_{set}$. 
Since $set$ is finite, $T_{set}$ also is  finite.  
Moreover,   $T_{set}$ cannot grow because any  transaction $T'$ added to
the array $\mathit{STATE}[1..n]$ has  a larger help date than $T$
(such a transaction $T'$ has  asked for help after $T$
and  due to the  ${\sf Fetch\&Increment}()$  operation the  help  dates 
are  monotonically increasing).  
So  to complete  the contradiction  we  need to  show that 
(a) $P_y$  is never blocked forever in an internal while loop 
(Figure~\ref{fig:main-construction})
and (b) eventually $T_{set} = \emptyset$. 

If $T_{set}$  is not empty, ${\sf  select\_next\_process}()$  returns
the process identity $j$ for some committed transaction $T'\in T_{set}$. 
On line  \ref{BA09}, the processor 
$P_y$ will see $T'$  in the list and perform $committed \gets true$. 
Hence, $P_y$ cannot block forever in an internal while loop. 
Then, on line \ref{BA32}, $P_y$  updates  $\mathit{LAST\_CMT}[y,j]$.
Let  us   observe  that,  during  the   next   iteration   of  
${\sf select\_next\_process}()$ by $P_x$,  $T'$ is not be added to  $set$ 
(line  \ref{CA01}) and, consequently, 
there is then  one less  transaction in $T_{set}$. And this
continues until $T_{set}$ is empty.  After this occurs,   
each time processor $P_y$ invokes ${\sf  select\_next\_process}()$,  it
obtains the process identity $i$, which invalidates the contradiction 
assumption and proves the lemma. 
\renewcommand{\toto}{lemma:help-commit-1}
\end{proofL}
%---------------------------


\begin{lemma}
\label{lemma:helped-transaction}
Any invocation of a transaction $T$ that requests help 
(hence it has $helpdate \neq \infty$) is eventually committed.
\end{lemma}

\begin{proofL}
Let us first observe that all transactions that require help 
have bounded and different help dates  (lines \ref{BA36}-\ref{BA37}, 
 \ref{CA11}-\ref{CA12}  or \ref{DA05}-\ref{DA06}).  
Moreover,  once defined, the helping date for a transaction is not modified. 

Among all the transactions that  have not been committed and require help,
let  $T$  be the transaction with the smallest help date. 
Assume that  $T$ has been issued  by  process  $p_i$ owned  by processor
$P_x$  (hence,  $P_x$ has required  help for  $T$).  
Let us assume that $T$ is never committed. The proof  is by  contradiction. 


As $T$ has the smallest help date, it follows from 
Lemma~\ref{lemma:help-commit-1}  
that there is a time after  which all the processors that call 
${\sf select\_next\_process}()$  obtains the process identity $i$.
%(lines \ref{CA01}, \ref{CA02}, \ref{CA04} and \ref{CA18}). 
Let $\cal P$ be this non-empty set of processors. 
(The other processors are looping  in  a while loop or are slow.) 
Consequently, given that all transactions that are not slow are trying 
to commit  $T$ (by performing  a ${\sf compare\&swap}()$  to add it  to the
list), that the list is not modified anywhere else, and that we assume 
that $T$ never commits, there is a  finite time after which the descriptor 
list does no longer  increase. Hence, as  the  predicate  $(\downarrow
current).next=\bot$ becomes eventually true,  we conclude that at least one  
processor $P_y \in \cal P$ cannot 
be  blocked  forever in a  while  loop. Because the list is no
longer  changing, the predicate of line  \ref{BA25} then becomes 
satisfied at $P_y$.
It follows that,  when  the processors  of  $\cal P$ execute line \ref{BA28}, 
eventually one of them successfully executes the compare\&swap that commits 
the transaction $T$ which contradicts the initial assumption. 

As the helping dates are monotonically increasing, it follows that
any  transaction $T$ that requires help is eventually committed. 
\renewcommand{\toto}{lemma:helped-transaction}
\end{proofL}
%---------------------------


%---------------------------
\begin{lemma}
\label{lemma:no-infinite-loop}
No processor $P_x$ loops forever in an internal while loop
(lines \ref{BA06}-\ref{BA11} or  \ref{BA19}-\ref{BA24}). 
\end{lemma}

\begin{proofL} 
The proof is by contradiction.
Let $P_y$ be a processor that loops  forever in an internal while loop. 
Let $i$ be the process identity it has obtained from its last call to 
${\sf select\_next\_process}()$ (line  \ref{BA02}) and $P_x$  be
the processor owner of $p_i$. 

Let us first show that  processor $P_x$ cannot loop forever in an internal
 while loop. Let us assume the contrary. 
Because processor $P_x$ loops forever we never have $((\downarrow current).next=\bot)~\vee~
\mathit{committed}$, but each time it executes the loop body, 
$P_x$ invokes ${\sf prevent\_endless\_looping}(i)$ (at line \ref{BA07} 
or \ref{BA20}). The code of this procedure is described in 
Figure~\ref{fig-prevent-looping}.  
As $i \in \mathit{OWNED\_BY}[i]$ and $P_x$ invokes infinitely often 
${\sf prevent\_endless\_looping}(i)$, it follows from lines 
\ref{DA01}-\ref{DA03}  and the current value of  $my\_last\_cmt$ (that 
points  to the last committed transaction issued by a process owned by 
$P_x$, see line~\ref{BA44}) that 
$P_x$'s local variable $k2\_counter$ is increased infinitely often.
Hence, eventually this number of invocations attains $K2$. 
When this  occurs, if not yet done, $P_x$ requires help  for the 
transaction issued by $p_i$ (lines \ref{DA03}-\ref{DA06}). 
It then follows from  Lemma \ref{lemma:helped-transaction},  
that $p_i$'s  transaction $T$  is  eventually  committed. As the  pointer
$current$ of $P_x$ never skips a descriptor of the list
and the list contains all and only committed transactions, 
we eventually have $(\downarrow current).tid = \langle i,i\_tr\_sn \rangle$
(where $i\_tr\_sn$  is $T$'s sequence number among the transactions issued
by $p_i$). When this occurs,  $P_x$'s local variable  $\mathit{committed}$ 
is set to $\mathit{true}$ and $P_x$ stops looping in an internal while loop. 


Let us now consider the case of a processor  $P_y \neq P_x$. 
Let us first notice that the only way for $P_y$ to execute $T$ is when 
$T$   has   requested   help    (line   \ref{CA01}   of   operation   ${\sf
select\_next\_process}()$). 
The proof follows from the fact that, due to
 Lemma~\ref{lemma:helped-transaction}, $T$ is eventually committed. 
As previously (but now  $current$ is  $P_y$'s local variable),
the predicate  $(\downarrow current).tid = \langle i,i\_tr\_sn \rangle$ 
eventually becomes true and processor $P_y$ sets  $\mathit{committed}$ 
 to $\mathit{true}$. $P_y$ then stops looping inside an internal while 
loop (line \ref{BA08} or \ref{BA22})  which concludes the proof of the lemma.
\renewcommand{\toto}{lemma:no-infinite-loop}
\end{proofL}

%---------------------------
\begin{lemma}
\label{lemma:transactions-are-committed}
Any invocation of a transaction $T$ by a process  is eventually committed.
\end{lemma}

\begin{proofL} 
Considering  a processor $P_x$,   let $i\in \mathit{OWNED\_BY}[x]$ be 
the current value of its local control variable $my\_next\_proc$.  
Let $T$ be the current transaction issued by $p_i$. 
We first show that $T$ is eventually committed.

Let us first observe that, as $p_i$ has issued $T$, $P_x$ has 
executed line \ref{BA42} 
where it has updated $\mathit{STATE}[i]$ that now refers to that transaction. 
If  $P_x$  requires help for $T$, the result  follows  from 
Lemma~\ref{lemma:helped-transaction}. 
 Hence,  to show that $T$ is eventually committed, we show that, if  $P_x$
does not  succeed in committing  $T$ without help, it necessarily  requires
help for it.  This  follows from the code of the procedure 
${\sf select\_next\_proc}()$. There are two cases. 
\begin{itemize}
%\vspace{-0.2cm}
\item
${\sf select\_next\_process}()$ returns $i$. 
In that  case, as $P_x$ does not loop forever in a while loop
(Lemma~\ref{lemma:no-infinite-loop}), it eventually executes lines 
\ref{BA33}-\ref{BA38} and consequently either commits $T$ or
 requires help for $T$ 
at line~\ref{BA37}. 
%
%\vspace{-0.2cm}
\item ${\sf select\_next\_process}()$ never returns $i$. 
In that case, as $P_x$ never loops forever in a  while loop 
(Lemma~\ref{lemma:no-infinite-loop}), it follows that 
it repeatedly invokes ${\sf select\_next\_process}()$ and, 
as these invocations do not return $i$, the counter $k1\_counter$ 
repeatedly increases and eventually attains the value $K1$. When 
this occurs $P_x$ requires help for $T$ (lines \ref{CA07}-\ref{CA15}) 
and, due to Lemma \ref{lemma:helped-transaction}, $T$ is eventually committed. 
\end{itemize}


Let  us  now  observe that  that, after  $T$  has been committed  (by  some
processor), $P_x$ executes lines~\ref{BA39}-\ref{BA44}  where  it
proceeds to the  simulation of its next process  (as defined by 
${\sf select}(\mathit{OWNED\_BY}[x])$).  It then  follows 
from the previous reasoning  that  the next transaction of the  process 
that is selected (whose identity is kept in $my\_next\_proc$) 
is eventually committed. 

Finally, as the function ${\sf select}()$ is  fair,  it follows that 
no  process  is  missed  forever and, consequently, any transaction 
invocation  issued by a process is eventually committed. 
%\vspace{-0.5cm}
\renewcommand{\toto}{lemma:transactions-are-committed}
\end{proofL}
%---------------------------



%---------------------------
\begin{lemma}
\label{lemma:committed-once}
Any invocation of a transaction $T$ by a process is  committed at most once.
\end{lemma}

\begin{proofL}
Let $T$ be a transaction committed by a processor $P_y$ (i.e., 
the corresponding ${\sf Compare\&Swap}()$ at line \ref{BA28} is successful). 
$T$ is identified $\langle  i, \mathit{STATE}[i].ts\_sn \rangle$.
As  $P_y$ commits  $T$,  we conclude that $P_y$ has previously 
executed lines \ref{BA06}-\ref{BA28}. 
\begin{itemize}
%\vspace{-0.2cm}
\item 
We conclude from  the last update of $\mathit{STATE}[i].last\_ptr=pt$ by $P_y$
(line \ref{BA42}) and  the fact that $P_y$'s $current$ local variable is 
initialized  to   $\mathit{STATE}[i].last\_ptr$,  that  $T$ is not in the 
descriptor list before the transaction pointed to by $pt$. 
%\vspace{-0.2cm}
\item 
Let us consider the other part of the list. 
As $T$ is committed by $P_y$, its pointer  $current$  progresses from 
 $\mathit{STATE}[i].last\_ptr=pt$ until its last value that is  such that
$(\downarrow current).next=\bot$.  It then follows from 
lines \ref{BA08} and  \ref{BA22} that  $P_y$ has never 
encountered a  transaction identified  
$\langle  i, \mathit{STATE}[i].ts\_sn \rangle$ (i.e., $T$) 
while traversing the descriptor list. 
\end{itemize}
It follows from the two previous observations  that, 
when it  is committed (added to  the list), transaction $T$ was not already
in the list, which concludes the proof of  the lemma. 
\renewcommand{\toto}{lemma:committed-once}
\end{proofL}
%---------------------------

%---------------------------
\begin{lemma}
\label{lemma:trans-code}
Each invocation of a transaction $T$ by a process  is committed exactly once.
\end{lemma}

\begin{proofL}
The proof follows directly from Lemma \ref{lemma:transactions-are-committed}
and Lemma \ref{lemma:committed-once}.
\renewcommand{\toto}{lemma:trans-code}
\end{proofL}




%---------------------------
\begin{lemma}
\label{lemma:non-trans-code}
Each invocation of non-transactional code issued by a process is
executed exactly once.
\end{lemma}

\begin{proofL}
This lemma  follows directly  from lines \ref{BA39}-\ref{BA44}:  once the
non-transactional  code separating two  transaction invocations  has been
executed, the processor $P_x$ that owns the corresponding process $p_i$ 
makes it progress to the beginning of its  next transaction (if any). 
\renewcommand{\toto}{lemma:non-trans-code}
\end{proofL}
%---------------------------



%---------------------------
\begin{lemma}
\label{lemma:process-progress}
The simulation is starvation-free (no process is blocked forever
by the processors).
\end{lemma}

\begin{proofL}
This follows directly from Lemma \ref{lemma:no-infinite-loop},
Lemma \ref{lemma:trans-code}, Lemma \ref{lemma:non-trans-code}
and the definition of the function ${\sf select}()$. 
\renewcommand{\toto}{lemma:process-progress}
\end{proofL}
%---------------------------




%---------------------------
\begin{lemma}
\label{lemma:trans-linearizability}
The transaction invocations issued by the processes are linearizable. 
\end{lemma}

\begin{proofL}
To prove the lemma we have (a) to associate a linearization point  with 
each  transaction  invocation, and (b) show that the corresponding sequence
of linearization points is consistent, i.e., the values read from  
$t$-objects by a transaction  invocation $T$ are uptodate  
(there have not been overwritten). 
As far as  item (a) is concerned, the  linearization  point of 
a transaction invocation is defined as follows\footnote{The 
fact  that a  transaction invocation is {\it read-only} or {\it  update} 
cannot always be statically determined. It can depend on the 
code of transaction (this occurs for example when a transaction  
behavior depends on  a predicate on values read from $t$-objects).
In our case, a read-only transaction   is a transaction with an empty write
set (which cannot be always statically determined   by a compiler).}.
\begin{itemize}
%\vspace{-0.2cm}
\item Update transactions 
(these are the transactions that write at least one $t$-object). 
The linearization point of the invocation of an update transaction is the time 
instant of the (successful) compare\&swap statement that entails its commit.  
%\vspace{-0.2cm}
\item Read-only transactions. 
Let $W$  be the set of update  transactions that have written  a value that
has been read 
by the considered read-only transaction. Let $\tau_1$ be the time just after
the maximum linearization point of the invocations of the transactions in $W$ 
and $\tau_2$ be the time at which the first execution 
of the considered transaction has started.
The linearization point of the transaction is then $\max(\tau_1,\tau_2)$.
%The linearization point  of the invocation of a read-only transaction is 
%the time instant of its last read of a $t$-object (by the processor $P_y$ 
%that commits it).
%
\end{itemize}

To prove item (b) let us consider the order in which  the transaction 
invocations are added to the descriptor list (pointed to by $\mathit{FIRST}$). 
As we are about to see, this  list and the linearization order are not
necessarily the same for read-only transaction invocations.  
Let us observe that, due  to the atomicity of the compare\&swap statement, 
a single transaction  invocation at a time is added to the list.

Initially, the  list  contains a single fictitious  transaction  that gives
an initial value to every  $t$-object. 
Let us assume that the linearization order of all the transaction invocations 
that have been committed  so far (hence they define the descriptor list) is 
consistent  (let us observe that this is initially true). 
Let us consider  the next transaction $T$ that is committed (i.e., added to
the list). 
As previously, we consider two cases.  Let $p_i$ be the process that issued
$T$,  $P_x$ the  processor that  owns $p_i$  and $P_y$  the  processor that
commits $T$.    
%
\begin{itemize}
%\vspace{-0.2cm}
\item 
The transaction is an update transaction (hence, $ws\neq \emptyset$). 
In that case, $P_y$ has found $(\downarrow current).next$ $=\bot$
(because  the compare\&sap  succeeds) and  at line~\ref{BA25}, just before
committing,  the predicate  $\neg committed \wedge \neg overwritten$ 
is satisfied.  

As  $overwritten$ is false, it follows that none of the values read by $T$ 
has been overwritten. Hence,  the reads and writes on 
$t$-objects issued by $T$ can appear as having been executed atomically at 
the time of the compare\&swap. Moreover,  the values of the $t$-objects 
modified by $T$  are saved in the  descriptor attached to the list
by the compare\&swap and the global state of the $t$-objects is 
consistent (i.e., if not overwritten  before, any future read of any of these 
$t$-objects obtains the value  written by $T$). 

Let us now consider the local state of $p_i$ (the process that issued $T$). 
There are  two cases. 
\begin{itemize}
%\vspace{-0.2cm}
\item $P_x=P_y$ (the transaction is committed by the owner of $p_i$). 
In that case, the local state of $p_i$ after the execution of $T$ 
is kept in $P_x$'s local variable $i\_local\_state$ (line \ref{BA15}). 
After processor $P_x$ has executed the 
non-transactional code that follows the invocation of $T$ (if any, line
\ref{BA39}), it  updates $\mathit{STATE}[i].local\_state$ with 
the current value of  $i\_local\_state$ (if $p_i$ had not yet terminated,
line~\ref{BA42}). 
%
%\vspace{-0.2cm}
\item $P_x \neq P_y$ (the processor that commits $T$ and the owner of $p_i$ 
are different processors). 
In that case, $P_y$ has saved the new local state of $p_i$ in 
$\mathit{DESCR}.local\_state$ (line \ref{BA27}) just before 
appending $\mathit{DESCR}$ at the end of the descriptor list. 

Next, thanks to the the predicate $i \in \mathit{OWNED\_BY}[x]$
in the definition of $set$ at line \ref{CA01}, 
there is an invocation of  ${\sf select\_next\_process}()$ by  $P_x$ 
that returns $i$. When this occurs,  $P_x$  discovers at line~\ref{BA09} 
or \ref{BA22} that the transaction $T$ has been  committed by another 
processor. It then retrieves the local state of $p_i$ (after 
execution of  $T$) in $(\downarrow current).local\_state$, 
saves it in $i\_local\_state$ and (as in the previous item)  eventually 
writes it  in  $\mathit{STATE}[i].local\_state$ (line \ref{BA42}).
\end{itemize}
It follows that, in both cases,  the value saved in 
$\mathit{STATE}[i].local\_state$  is the local state of $p_i$ after the 
execution of $T$ and the non-transactional code that follows $T$ (if any). 
%
\item 
The transaction is a read-only transaction (hence, $ws=\emptyset$).
%
In that case, $T$ has not modified the state of the $t$-objects. 
Hence,  we only have    to prove   that the  new  local state  of $p_i$  is
appropriately updated and saved in  $\mathit{STATE}[i].local\_state$.

The proof is the same as for the case of an update transaction. 
The only difference lies in the fact that now it is possible to have 
$overwritten \wedge ws = 0$.  If  $overwritten$ is true,  $T$ can no 
longer be linearized  at the commit point. That is why it is 
linearization point has been defined 
just after the maximum linearization point of the transactions it reads 
from (or the start of $T$ if it happens later), which makes it linearizable.  
\end{itemize}
%\vspace{-0.4cm}
\renewcommand{\toto}{lemma:trans-linearizability}
\end{proofL}
%---------------------------

\begin{lemma}
\label{lemma:all-linearizability}
The   simulation  of  a  transaction-based $n$-process  program   
by  $m$  processors (executing the algorithms 
described in Figures \ref{fig:initialization}-\ref{fig-prevent-looping})
is linearizable. 
\end{lemma}

\begin{proofL}
Let us first observe that, due to Lemma \ref{lemma:trans-linearizability},
The transaction invocations issued  by the processes are linearizable, from
which we conclude that the set of $t$-objects (considered as a single 
concurrent object $TO$) is linearizable. 
Moreover, by  definition, every $nt$-object is linearizable. 


As (a)linearizability is a {\it local} consistency property 
\cite{HW90}\footnote{A  property $P$  is  local if  the  set of  concurrent
objects (considered as a single object)
satisfies $P$  whenever each object taken alone  satisfies $P$. It is
proved in \cite{HW90}  that linearizability is a local  property.} and (b) 
 $TO$  is linearizable and  every $nt$-object  is linearizable,  it  follows
that the  execution of the multiprocess program is linearizable.  
\renewcommand{\toto}{lemma:all-linearizability}
\end{proofL}
%---------------------------




\begin{theorem}
\label{theorem:main}
Let $\mathit{PROG}$ be a transaction-based $n$-process program. 
Any simulation of $\mathit{PROG}$ by $m$ processors executing the algorithms 
described in Figures \ref{fig:initialization}-\ref{fig-prevent-looping}
is an execution of  $\mathit{PROG}$. 
\end{theorem}

\begin{proofT}
A formal statement of this proof  requires an heavy  formalism. Hence we 
only give a sketch of it. Basically, the  proof follows from 
Lemma~\ref{lemma:process-progress} and Lemma~\ref{lemma:all-linearizability}. 
The execution of $\mathit{PROG}$ is obtained by projecting  the execution 
of each  processor on the simulation of the transactions it commits and 
the execution of the non-transactional code of each process it owns. 
\renewcommand{\toto}{theorem:main}
\end{proofT}
%---------------------------



%====================================================================
\section{The number of tries is bounded}
\label{sec:bounded-tries}
This section presents a bound  for the maximum number of times a transaction
can  be unsuccessfully executed by a  processor before  being  committed,
namely, $O(m^2)$. A  workload that has this bound is then given. 


\begin{lemma}
\label{lemma:1max} 
At any time and  for any processor $P_x$, there is at most one atomic register 
$\mathit{STATE}[i]$ with  $i\in \mathit{OWNED\_BY}[x]$ such that
the corresponding transaction 
(the identity of which is $\langle i, \mathit{STATE}[i].tr\_sn\rangle$)
 is not committed and $\mathit{STATE}[i].help\_date\neq +\infty$. 
\end{lemma}
%\vspace{-0.2cm}

\begin{proofL}
Let us first notice  that the  help date of a transaction invoked by a 
process $p_i$ can be set to  a finite value  only by the  processor $P_x$ 
that owns $p_i$.  There are two places where $P_x$  can request help. 
\begin{itemize}
%\vspace{-0.2cm}
\item 
This first location is in  the ${\sf prevent\_endless\_looping}()$ procedure.
In that  case, the  transaction  for which  help  is required is  the last
transaction invoked by  process $p_{my\_next\_proc}$. 
%\vspace{-0.2cm}
\item 
The second  location is on line  \ref{BA37} after the  transaction 
invocation  $T$ aborts. 
It follows from line \ref{CA03} of  the operation ${\sf  select\_next\_process}()$   
that  this invocation is also from  the last
transaction invoked by  process $p_{my\_next\_proc}$. 
\end{itemize}
So  we  only  need  to  show  that $my\_next\_proc$  only  changes  when  a
transaction is committed, which  follows directly from the predicates at lines 
 \ref{BA34} and  \ref{BA35} and the statements of line  \ref{BA44}. 
\renewcommand{\toto}{lemma:1max}
\end{proofL}   

\begin{theorem}
\label{proof:try-bounds}
A transaction $T$  invoked by a process $p_i$ owned by processor  $P_x$ 
is tried unsuccessfully  at most $O(m^2)$  times  before being  committed.
\end{theorem}


\begin{proofT}
Let  us first   observe  that a  transaction  $T$  (invoked by  a process
$p_i$)  is executed once before its  help date is  set  to a finite value
(if it is not committed after that execution).
This is  because only the owner $P_x$ of $p_i$ can  select $T$
(line \ref{CA03}) when its help date is  $+\infty$.
Then, after it has  executed $T$ unsuccessfully once, $P_x$  requests
help for $T$ by setting its  help date to a finite value (line \ref{BA37}).

Let us now compute   how many  times $T$  can  be  executed unsuccessfully
(i.e.,  without being committed) after its help date has been  set to a
finite value.
As there  are  $m$   processors and all  are equal (as far as  helping is
concerned),   some  processor must execute $T$  more  than $O(m)$ times
in  order for  $T$  to be  executed more  than $O(m^2)$ times.
We show that this is impossible. More precisely,
assuming  a processor $P$ executes  $T$, there are $3$ cases that can cause
this execution to be unsuccessful  and as shown below
each case can cause at most $O(m)$ aborts of $T$ at $P$.
%
\begin{itemize}

\item
Case 1. The first case  is that some other transaction $T1$ that does
not  request help  (its help date is $+\infty$)  is committed  by some  other
processor $P2$ causing $P$'s execution of $T$ to abort.
Now by lines  \ref{CA02} and \ref{CA03} after $P2$  commits $T1$, $P2$ will
only  be executing  uncommitted transactions  from the  $\mathit{STATE}$
 array with
finite help dates at least until  $T$ is committed, so any subsequent abort
of $T$ caused  by $P2$ cannot be caused by  $P2$ committing a transaction
with $+\infty$ help date.
So the maximum  number of times this  type of abort can happen  from $P$ is
$O(1)$.

\item
Case  2. The second case  is when  some  other uncommitted transaction
$T1$ in  the $\mathit{STATE}$  array with a  finite help  date is committed
by some other processor $P2$ causing $T$ to abort.
First by  lemma \ref{lemma:1max} we know  that there is a  maximum of $m-1$
transactions that are not $T$ that  can be requesting help at this time and
in order for them  to commit before $T$ they must have  a help date smaller
than $T$'s.
Also  by lemma  \ref{lemma:committed-once} we  know that  a  transaction is
committed exactly once  so this conflict between $T1$  and $T$ cannot occur
again at $P2$.
Now after committing  $T1$, the next transaction (that asks  for help) of a
process that  is owned by  the same processor  that owned $T1$ will  have a
larger help  date than $T$  so now there  are only $m-1$  transactions that
need help that could conflict with $T$.
Repeating this we have at most $O(m)$ conflicts of this type for $P$.
%\vspace{-0.2cm}
\item
Case  3. The  third case is  that $P$'s  execution of $T$  is aborted
because some other process has already committed $T$.
Then on  line \ref{BA08} $P$ will see  that $T$ has been  committed and not
execute it again, so we have at most $O(1)$ conflicts of this type.
\end{itemize}
%\vspace{-0.6cm}
\renewcommand{\toto}{proof:try-bounds}
\end{proofT}




\paragraph{The bound is tight}
The  execution that is described below shows that a  transaction $T$ 
can be tried  $O(m^2)$ times before being committed. 

Let  $T$ be a transaction owned by processor $P(1)$ such that 
$P(1)$ executes $T$ unsuccessfully once and requires help 
by setting its help date to a finite value. 
Now, let us assume that each of the  $m-1$ other processors  is 
executing a transaction it owns,  all  these  transactions
conflict with  $T$ and  there are no  other uncommitted  transactions with
their help date set to a finite value.
 
Now $P(1)$ starts executing $T$  again, but meanwhile processor $P(2)$ 
commits its own transaction which causes $T$ to abort. 
Next $P(1)$  and $P(2)$  each try to execute $T$, but meanwhile processor $P(3)$  
commits its own transaction causing $P(1)$ and $P(2)$ to abort $T$. 
Next $P(1)$, $P(2)$, and $P(3)$ each try execute $T$, but meanwhile processor 
$P(4)$ commits its owns transaction causing $P(1)$, $P(2)$, and $P(3)$ 
to abort  $T$. Etc.  until  processor  $P(m-1)$  aborts all  the  
execution of $T$ by  other processors, resulting in all $m$ processor 
executing $T$. The transaction $T$ is then  necessarily committed by one of 
these final executions. 
So we have $1+1+2+3+\ldots+(m-1)+m$ trials of $T$ which is  $O(m^2)$.

%=============================================================================


\section{Conclusion}
\label{sec:conclusion}


%=============================================================================

\subsection{A short discussion}
\label{sec:discussion} 

The aim of the universal construction that has been presented 
was to demonstrate and investigate this type of construction for
transaction-based multiprocess programs. (Efficiency issues 
would deserve  a separate investigation.) To  conclude, we list  here a few
additional noteworthy properties of the proposed  construction. 

\begin{itemize}

\item  The construction is for the family of transaction-based concurrent
programs  that are time-free (i.e., the semantics of which does not depend 
on real-time constraints). 

\item 
The construction  is lock-free and  works whatever the concurrency pattern
(i.e., it does not require concurrency-related  assumption such as 
obstruction-freedom). It  works for  both finite  and  infinite
computations and does not require  specific scheduling assumptions.  
%
Moreover,  it is  independent of the fact that 
processes  are  transaction-free (they then share  only
$nt$-objects),  do not have  non-transactional code (they then  share 
only $t$-objects  accessed by transactions)  or have both  transactions and
non-transactional code.  

\item  
The helping mechanism can be improved by allowing a processor to 
require help for a transaction only when  some condition is satisfied. 
These  conditions could be general or application-dependent. They could 
be static or dynamic and  be  defined in relation with an underlying
scheduler or a contention manager. 
The construction can also be  adapted  to benefit from an underlying 
scheduling allowing the owner of a process to be  dynamically defined.  

It could also be adapted to  take into account {\it irrevocable} transactions
\cite{SSDMS08,WSA08}. 
Irrevocability is an implementation property which can be 
demanded by the user for  some of its transactions. It states that 
the corresponding transaction cannot be aborted (this can be useful when 
one wants to include inputs/outputs inside a transaction; notice that, 
in our model, inputs/outputs appear in non-transactional code). 

\item  
We have considered a failure-free system. It is easy to see that, in a
 crash-prone system, the crash of a processor entails only the
crash of the processes it owns.  The processes owned by the processors that
do not crash are not prevented from executing.
Furthermore due to the helping mechanism, once a process has asked
for help with a transaction that transaction is guaranteed to commit
as long as there exsits at least one live failure free process.
\end{itemize}

In addition to the previous properties, 
the proposed  construction helps better understand the atomicity feature  
offered  by STM systems to users in order to cope with concurrency issues. 
Interestingly this construction has some ``similarities'' with 
general constructions proposed to cope with the net effect of 
asynchrony, concurrency and failures, such as 
the BG simulation \cite{BG93}  (where there are simulators that execute 
processes) and   Herlihy's universal construction to  build  wait-free 
objects \cite{H91} (where an underlying list of consensus objects 
used to represent the state of the constructed object lies at the 
core of the construction). The study of these  similarities would 
deserve a  deeper investigation.

The previous chapter explored an area of transactional memory
research that focuses on improving STM protocols without effecting
how the user interacts with the STM, that is by ensuring some implementation
level properties or by increasing performance.
This chapter, while similiar to the previous chapter in suggesting propertes
and showing how a protocol can implement them,
takes a more visible approch
that directly effect the interaction between the programmer and
the STM.
Abstractly, it examines how the semantics are defined between
the programmer and the STM protocol and suggests they be simplified.
Previous research has expected some level of interaction
between the programmer and aborted transactions, while this
chapter suggests the notion of commit/abort be completely abstracted
away from the programmer level left to be soely an implemntation
concern.
This frees the programmer from having to consider if his transacion
might not commit and to eiter try to prevent such a situation, or to 
come up with ways to deal with it when it does.

As a final motivation for thse simplifited semantics, let us consider how it compares
to the consistency condition of opacity.
Opacity differs from linearizability or serializibility in that
it frees the programmer from having to worry about consistency
issuses that could arrise in aborted transactions.
This liveness suggessted in this chapter for STM protocols differes
from previous suggestions in that it frees the programmer from
having to worry that his transaction might not commit.
In a way it can be considered the equivalent to opacity except
opacity considers correctness (bad things happening in aborted transactions),
while this chapter considers liveness (a transaction that is only aborted).

Without opacity the programmer has to come up with solutions in order
to prevent things like such as invalid pointers, infinate loops, or divide by $0$ errors
from happening in aborted transactions.
Without the liveness suggested in this chapter a programmer has to come
up with solutions in the case that a transaction is not able to progress
due to the actions of some other process in the system.

Further research is still needed on how the semantics of a transaction
can be simplified.
For example currently transactions are only able to contain reads
and writes, while things that input and output are not allowed,
and the interaction between transactions and other synchronization
methods is undefined or prohibited.