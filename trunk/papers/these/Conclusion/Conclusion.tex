% \section{Conclusions and future work}
Software transactional memory is a attractive alternative to locks for writing concurrent programs
due to the fact that it abstracts away many of the difficulties.
A programmer using STM can simply create blocks of code surrounded by the $\act{atomic}$
keyword which are then guaranteed by the underlying STM implementation to appear
as being executed atomically,
thus preventing the programmer from worrying about complex synchronization details
and avoiding problems such as deadlock.
Unfortunately in its current form, using STM is not quite so simple.
The reason for this is that are many more details a programmer must consider when using transactions
further than just defining atomic blocks.
Part of the reason for this is that STM exists as a single component of a more complicated
system of concurrency for which many details must be considered or assumptions made.
In fact, many of the not so subtle-subtleties
of the STM abstraction are actively debated by the research community.

Given this complex system where even experts have difficulty in coming us with precise
and clear ways of defining and dealing with
subtleties this thesis takes a different approach.
% Instead of trying to come up with a precise and complete definition for transactions and how they
% fit in a concurrent system, this thesis approaches the problem in a different way.
The key of this approach is that this thesis takes as monumental importance
the view that STM is a high level abstraction whose
primary goal is to make concurrent programming easier.
All other goals such as performance are secondary.
In order to do this, the thesis starts with a high level goal for defining the STM
system focused on ease of use (from Definition \ref{def:stm-def}):
``\emph{STM is an abstraction designed to simplify concurrent programming that exists as a component of a larger system,
but the semantics of the STM abstraction and how it interacts with the rest of the system must be clear and easy to understand and use,
otherwise simplicity is lost.}''
Which is then followed by definition for
a transaction aiming to satisfy this goal avoiding specific system details (from Definition \ref{def:trans-def}):
``\emph{A transaction is any block of
code defined by the programmer to be executed by a given process.
This transaction is executed atomically exactly once
taking in consideration all parts of the system.}''


Given that this definition gives no details on the complex interactions of transactions in
the system, instead of trying to expand this definition with further details, this
thesis looks at ways to come up with solutions that fit within the confines of this
definitions that promote ease of use.
Of course precise definitions of semantics and properties must then be used as solutions
for the specific problems, but the goal is that the high and low levels can coexist.
A programmer who
understands the high level definition should find the specific details of the low level definitions as a
natural extension.

% hen from this definition this thesis looks at some of the complexities that exist with STM
% and attempts to come up with precise ways to deal with them in order that they do not violate
% our high level definitions.

In doing this, the thesis was broken up into four main chapters of content, with each chapter
introducing a general area of STM research 
focused on ease of use
before tacking a specific problem in this area.

\paragraph{Overview}
The first chapter took a common view at what is widely viewed as the most important concept
for STM to ensure.
That is the idea that transactions should be executed atomically with respect to each other
with aborted transactions seeing no invalid states of memory, following the basic
idea of consistency criterion such as opacity or virtual world consistency.
In this area of STM research protocols and properties are explored while
ensure that the consistency criterion satisfied, placing consistency as a first priority.
Specifically this chapter encourages exploring multiple properties at once, with
the combination of read invisibility and permissiveness looked at in conjunction
with the correctness criterion of opacity and virtual world consistency.
Looking at further combinations is also suggested.

The second chapter took the view that simply ensuring a consistency criterion
is not enough to fit in the high level transaction definition and instead
suggests to simplify the traditional view of transaction semantics.
That is, to ensure that every transaction is committed exactly once no matter
the concurrency patter of other processors in the system, effectively allowing
the programmer not have to worry that his transactions might not be executed;
in a way abstracting away the concept of aborts when considering progress.
As suggested in the chapter, further research could be conducted on the simplification
of and increased abstraction level of transactions, with an important problem being nested transactions.

The third chapter took the view that STM exists as an integrated part
of the users program and the concurrent system as a whole, thus
suggesting the viewing of STM as its own separate entity in a concurrent system does not fit
in our high level definition.
In order to integrate STM in the system the details and semantics of how a transaction
fits within a system must be defined.
Specifically this chapter looked at the problem of concurrently accessing
shared memory inside and outside of a transaction, and suggested that in order
to follow our high level definition terminating strong isolation should be insured.
As future research, there are many other components in the system that should be considered,
such as how STM should interact with other synchronization methods
and with non-reversible actions such as system calls.


The fourth chapter looked at the concept of libraries in concurrent programing.
Libraries implementing common abstractions in a correct and efficient
manner are becoming an important and widely used part of a concurrent system.
Thus it is important to develop libraries for STM systems as well so that STM can become
a complete and easy to use integrated part of the concurrent system.
Given their importance in concurrent programs libraries can also become a bottleneck
in programs and due to the specific type of synchronization done in STM
libraries need to be developed specifically for use in transactions to avoid these bottlenecks.
Interestingly one of the ways this chapter suggests doing this is by using additional
operations in order to write these libraries that break the traditional TM interface, thus complicating things, seemingly
violating the spirit of this thesis.
From this, two important things should be taken away, first is that this thesis encourages
research on ways to improve the performance of the STM abstraction by extending it, but more importantly
the programmer should not be required to know the details of these operations as they should be left to be used by programmers
who know the intricacies of the STM system in order to implement things such as libraries.
In this chapter a binary tree library implementing the map abstraction is developed
with future work being suggested to design a more complete library of abstractions for use in STM.


\paragraph{Final discussion}
In a way this thesis contradicts itself, a chapter will suggest that an STM should
be designed in a certain way, while the following chapter will then introduce
a protocol that does not satisfy this design.
No chapter introduces a permissive, terminating strong isolating, STM protocol that ensures
every transaction is committed.
Instead each chapter focuses on a certain problem concerning the ease of use of STM
in specific.
Thus, important future work (notably motivated by the first chapter) is too look at all the desirable
properties together and see if the design of such a system is possible.


It should be noted that, in no way does this thesis suggest its definitions or results be the final
word on those topics in STM.
Instead, the opposite should be promoted:
After defining these first semantics focused on ease of use, the next step is to examine them,
find out what parts programmers use and understand and which parts are not used often, or
are too complex, or too difficult to implement, then refine them in an iterative process.
With the goal that someday programmers will be able to write efficient concurrent programs
(nearly!) as simply as they write sequential programs today.
